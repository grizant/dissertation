\chapter{Concluding remarks}\label{Chap:conclusion}
\indent\indent The final chapter of this dissertation serves as a summary of the work presented in the previous chapters.

% 
%In the first chapter of the dissertation, an overview of the process of quantitative risk assessment and the steps it entails was given. The dose-response modeling and quantitative risk estimation was addressed.  Here, risk $R(d)$ was defined as the probability of a subject exhibiting a pre-defined adverse effect when exposed to a particular dose level, $d$, of a hazardous agent. Two key assumptions were made on this fundamental definition of risk.  The first assumption was that risk is a monotone increasing function. The second assumption was that non-zero background risk at $d=0$ may exist in the population of interest. To correct for this spontaneous background risk, excess risks such as additional risk and extra risk were further introduced. These excess risks are more commonly used in quantitative risk estimation. After that, difficulties in low-dose risk estimation were discussed and a modern technique---benchmark dose analysis---was introduced. In benchmark dose analysis, one uses the functional specification for $R(d)$ to provide low-dose estimates for risk and/or excess risk. To estimate the benchmark dose (BMD) and the corresponding benchmark dose lower limit (BMDL) in quantal-response data setting (as is the focus of this dissertation), the traditional frequentist method through parametric maximum likelihood estimation was first introduced, and then a thorough literature review on established parametric hierarchical Bayesian benchmark dose analysis was provided.
%
%Chapter \ref{Chap:Repar} introduced 8 popular quantal-response models as seen in the U.S. EPA's BMDS software \citep{davi12}. These were forms of generalized linear models expressed via traditional $\beta-$type parameters. Because these parameters usually don't have pertinent risk analytic interpretations, it is difficult to construct prior distributions for them. As a result, objective, even improper, priors had to be used instead. In order to embrace useful prior information into the Bayesian hierarchical modeling, reparameterizations for these 8 models were performed by using the target quantity, BMD ($\xi$), background risk ($\gamma_0$) and in three-parameter models, also a risk at some dose level (usually the risk at the highest level, $\gamma_1$) as the new parameters. Mathematical forms have been derived; these reparameterizations present more burdensome notation for $R(d)$, however, allow benchmark analysts to formulate a clearer and more application-oriented hierarchical model, from which to produce inferences on BMD.
%
%The hierarchical Bayesian modeling framework was introduced in Chapter \ref{Chap:Bayesian Frame}. This included three major topics: prior specification, posterior approximation and Bayesian estimation. For positive quantity $\xi$, inverse gamma prior was specified by default; for probability quantities $\gamma_0$ and $\gamma_1$ (also for the risks at any other dose level), Beta prior was specified. Prior elicitation process using first quartile and median was introduced. When elicitation is not available, proper objective priors using $\xi\sim IG(0.001, 0.001)$ and $\gamma_0, \gamma_1 \sim Beta(\frac12, \frac12)$ were employed and recommended. Joint posterior density was approximated by using an adaptive Metropolis algorithm introduced in \citet{anth08}, convergence diagnosis and burn-in determination were applied mimicking a method introduced in \citet{gewe92}. Several decision theoretical Bayesian estimators were discussed and the posterior lower tercile estimator resulted from bi-linear asymmetric loss function was recommended for BMD point estimation. With the AM sample, 95\% BMDL was easily estimated as the 5th lower percentile of the AM sample. To investigate and assess the prior sensitivity on $\xi$, an $\epsilon-$contamination study followed by \citet[\S7.15]{ohag94} was introduced.
%
%The increasing number of available models for benchmark dose analysis brings in the issue of model uncertainty and adequacy. Chapter \ref{Chap:BMA} discussed this important issue. Due to the difficulty of finding reliable selecting criterion for model selection, a Bayesian model averaging (BMA) method introduced by \citet{homa99} was employed. In this method, those 8 quantal-response models were chosen as the uncertainty class and a mixture distribution for $\xi$ was constructed using the posterior model probabilities as the weights. The BMA BMD is not pursued while the BMA BMDL which satisfies the corresponding probability statement is approximated by using a `direct method' from each model-specific AM sample.
%
%In Chapter \ref{Chap:Example}, application of the proposed hierarchical Bayesian method was illustrated via a real carcinogenicity study performed by U.S. National Toxicology Program (NTP) on the chemical cumene. BMD and BMDL were first estimated by using the popular quantal linear model. To investigate prior sensitivity, three $\epsilon-$contaminating scenarios were considered, and it was shown that the estimates from quantal linear model was reasonably robust. After that, another popular model---logistic model---was fitted to the data and its estimates was found to be significantly different from quantal linear model's. Facing the issue of model uncertainty, the remaining 6 models were then fitted to the data and the Bayesian model averaged BMDL was produced. This showed an example that BMA adjustment could free the risk assessors from model inadequacies and inferential uncertainties which were frequently encountered when committing to a single-model benchmark analysis.
%
%Chapter \ref{Chap:Simu} aimed to evaluate the performance of the proposed method via a simulation study. 8 reparameterized quantal response models were fitted to a set of 2000 data sets generated by each of these models under 5 configurations and 3 sample-sizes. Asymptotic normality of the point estimates of BMD appeared to be satisfied when fitting the correct model to the data. The model-specific BMDL also appeared to converged to nominal as $N$ increased when the correct model was fitted to the data. When model were misspecified, however, both the point estimates and the BMDL became unreliable. The BMA BMDL, on the other hand, could provide much more stable and acceptable estimates.
%
%With the conclusion of this portion of the research, there are several natural extensions for further study. Mimicking previous Bayesian benchmark analysis methods \citep{shsm11,shsm12,shao12}, the current method assumed that the unknown parameters enter the joint prior independently. The correlation between parameters may not be ignored, however. If correlation between parameters can be elicited and successfully incorporated into the prior hierarchy, the BMD/BMDL estimation may be further improved. In addition, objective prior specification can be approached via many strategies when elicitation breaks down. The current method assumed independent $IG(0.001, 0.001)$ and $Beta(\frac12,\frac12)$ priors. Other forms may be pertinent, however, and these choices might still bring possible subjectivity to the proposed hierarchical model. It is also of interest to investigate how the proposed approach operates under different design configurations. The current research focused on a geometric, four-dose design, arguably the quintessential standard in cancer and laboratory-animal toxicology testing. Greater information might be gained about the pattern of dose response, however, and therefore about the BMD, if the number of doses is increased and/or the dose spacings are changed. Experimental design for dose-response studies with focus on the BMD is an emerging area in the statistical literature \citep{muri09,ober10,sand08,shsm12} and how to optimally design/allocate experimental resources for BMD estimation and inferences under a Bayesian paradigm is an open question.
