\chapter{Conclusion}\label{Chap:conclusion}

% \section{Summary}

%\indent\indent The final chapter of this dissertation serves as a summary of the work presented in the previous chapters. My goal is to become a pioneer and leader in the field of \emph{single-subject inference}, in which there are many open questions for further study. There is also substantial demand for these methods in the biomedical domain, within which I am passionate to make contributions. I plan to focus my future research in these areas as I develop my academic career.

\indent\indent The overarching goal of this collection of works is to gain insight for cellular RNA quantification from a single subject. When working in this setting, traditional statistical approaches are largely unavailable due to prohibitive sample size and lack of independent replication. This leads one to rely on informatic devices including knowledgebase integration (e.g., gene set annotations; Section \ref{sec:nof1pathways}) and external data sources (e.g. estimation of inter-gene correlation; Appendix \ref{App:C}). Common statistical themes include use of multivariate statistics (such as Mahalanobis distances and copulas) and computationally-extensive multiple significance tests. Notably, there is substantial demand for single-subject methods in the biomedical domain \citep{Bacchetti2011}. However, there is little statistical interest in studying data solely derived from a single subject at this time and thus there exists an opportunity for rapid advancement in methodology.

It was posited in Chapter \ref{Chap:Intro} that it is possible to derive context-relevant effect sizes and satisfactory significance testing operating characteristics (empirical size and power) for pathway testing of paired gene-expression measurements while accounting for the inter-observation correlation inherent to the \lq N-of-1\rq~setting. Indeed, the collective contribution of the three contained articles (Appendices \ref{App:A} -- \ref{App:F}) represents substantial progress towards meeting that goal. First, \citet{Schissler2015} introduced the Mahalanobis distance (MD) within pathways, which accounts for the observed correlation between paired samples (across genes) while providing a clinically-relevant effect size. These features represent a significant advance over the current state-of-the-art, N-of-1-\emph{pathways} Wilcoxon \citep{Gardeux2014}, as the latter did not account for inter-observation correlation and only offered a difficult to interpret, transformed P-value as a metric of differential pathway expression. Moreover, \citet{Schissler2015} showed that the MD methodology conferred more power to detect a simulated differentially expressed pathway (DEP) than the Wilcoxon approach.

In light of the potential detrimental impact \citep{Tamayo2016} of inter-gene correlation (as opposed to the correlation between the paired samples) on pathway testing false positive rates, I introduced the clustered-$T$ statistic (contained in Appendix \ref{App:C}, \ref{App:D}). This approach greatly outperformed the MD testing methodology given by \citet{Schissler2015} in simulated pathways with non-trivial inter-gene correlation. Notably, the MD effect size construction remains valid for the clustered-$T$ testing approach (refer to Chapter \ref{Chap:ct}). Therefore, the combined product of the two papers provide the aforementioned effect size and satisfactory testing characteristics as hypothesized. Finally, the aggregation framework of these paired statistics applied to single-cell transcriptomes of rare cells given by \citet{Schissler2016} provides both pathway differential effect size quantification and statistical testing in a small-cohort, noisy measurement setting This methodology provides a mechanistic glimpse at cell-cell heterogeneity that is not obtainable via conventional single-cell RNA-sequencing methodology that require many single-cell transcriptomes \citep{Ding2015, Grun2014}. Appendix \ref{App:eqns} provides a conceptual framework to unify the seemingly disparate approaches provided in the three articles.

%Below I provide future directions for the work contained in this dissertation.

% 
%\section{Other ongoing works}
%I am involved in several other projects that focus on refining, applying, and disseminating our methods. In brief, one project is working to improve the clustering step of kMEn (Section \ref{sec:kmen}; \cite{Li2016}). Another project is developing new methodology to predict asthma exacerbation by coupling our clinical assay virogram (\cite{Gardeux2015}) with N-of-1-\emph{pathways}. Further, we are releasing an \texttt{R} package for N-of-1-\emph{pathways} and the cell-cell distance analysis, and plan to submit a short application note describing its utility.
%
% \section{Future work}
%
%One future project could refine the ``clustered-$T$'' approach (Chapter \ref{Chap:ct} and Appendix \ref{App:C}). This method takes a frequenist approach while also injecting additional external information into the N-of-1-\emph{pathways} framework. It is natural when considering the incorporation of external information, however, to consider some form of Bayesian methodology (\cite{Christensen11}). Indeed, the inter-gene correlation structure could be modeled in a hierarchical fashion - providing a rich setting for differential pathway metric estimation and testing.
%
%Further still, a whole-transcriptome gene set testing approach in this Bayesian hierarchical model could be devised to ``borrow strength'' across pathways. (One could view genetic pathways as a spatially-correlated regions informed by the gene set ontological, topological, structure). However, this framework would likely be highly complex and, with the limited data on hand (two paired samples from one subject), subjective (prior) information and assumptions may heavily dominate the eventual inferences. Still, the flexibility, robustness, and predictive performance of a Bayesian framework shows promise to provide an effective tool for precision medicine.
%

%A second project would enhance the methodology presented in the aggregated cell-cell statistical distances article (Chapter \ref{Chap:ctcs} and Appendix \ref{App:E}). Briefly, the original framework includes calculating pathway differential expression between each pair of cells. Then a nonparametric, hierarchical bootstrap (a ``clustered'' bootstrap) was conducted to assess whether a pathway was differentially expressed between two groups of cells, sampled from patients. Two main issues could be explored here: First, single-cell RNA-sequencing is known to have much higher dropout rates (mRNA transcripts not amplified) than bulk RNA-seq. A simulation study of the effect of zero-inflation on pathway testing would be instrumental in determining the empirical pathway testing operating characteristics (size and power) and possibly lead to modifications of the framework (including modeling of dropout and amplified mRNA through mixture modeling). Second, by construction, the metrics generated from pairing all the cells are correlated. The underlying, induced covariance structure should be studied and potentially modeled to determine the multivariate characteristics of the pathway cell-cell distances. This could lead to interesting statistical insights and novel significance tests in the small-sample setting.
%

Future directions include extending my transcriptome analytics to proteomic data or to multiple samples (as opposed to simply paired). The transcriptome offers a glimpse at actual protein molecule activity within the cell, yet opinions differ on the correlation between the mRNA expression and the protein activity. Proteomics represent an ability to quantify protein abundance and states, but remain technically challenging and expensive. As proteomic technologies improve, in principle, many of our single-subject, gene-expression-based methods could be easily modified for detecting protein-level pathway differences. One could also develop analytics based on multiple samples from a single-patient. Hearkening back to classical N-of-1 studies in which a single measurement is tracked over time \citep{Nikles2015}, we could investigate single-subject, temporal pathway expression changes to make adaptive treatment decisions (one can envision some form of time-series approach). Or, for spatially-structured multiple samples (e.g., multiple samples from a tumor), we could generalize our distances metrics for higher dimensions.
