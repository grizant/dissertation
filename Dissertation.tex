%\documentclass[draft]{ua-thesis}
\documentclass[final]{ua-thesis}
\usepackage{multirow}
\usepackage[section] {placeins}
\usepackage{setspace}
%\onehalfspacing
%\doublespacing

% this package allows insertion of entire pdfs
\usepackage[final]{pdfpages}

%\linespread{1.75}
%\addtolength{\footskip}{24 pt}  % pushes page no. down at bottom of page

%\usepackage[breaklinks]{hyperref}
\usepackage{graphicx}
\usepackage{amsmath,amsfonts,amssymb}
\usepackage{amsthm}
%\usepackage{soul}  %for underline characters
%\usepackage{natbib}
%\bibliographystyle{plainnat}
%\bibpunct{(}{)}{;}{a}{,}{,}
\usepackage[backend=biber,style=authoryear,citestyle=authoryear,sorting=none]{biblatex}
\bibliography{/Users/grantschissler/Dropbox/bib/default,/Users/grantschissler/Dropbox/bib/library}
\defbibheading{refs}[refs]{%
\chapter*{References}%
}


\newtheorem{definition}{Definition}[section] %For definition and numbered in sections
\numberwithin{equation}{section}
%\numberwithin{table}{subsection}
%\numberwithin{figure}{subsection}

%\long\def\symbolfootnote[#1]#2{\begingroup% For special symbols in footnote
%\def\thefootnote{\fnsymbol{footnote}}\footnote[#1]{#2}\endgroup}

%\setlength{\topmargin}{0in}
%\setlength{\textheight}{8.5in}
%\setlength{\evensidemargin}{0in}
%\setlength{\oddsidemargin}{0in}
%\setlength{\textwidth}{6.5in}

\newcommand\solidrule[1][1cm]{\rule[0.5ex]{#1}{.4pt}}
\newcommand\dottedrule{\mbox{%
\solidrule[.3mm]\hspace{.7mm}\solidrule[.3mm]\hspace{.7mm}\solidrule[.3mm]\hspace{.7mm}\solidrule[.3mm]\hspace{.7mm}\solidrule[.3mm]\hspace{.7mm}%
\solidrule[.3mm]\hspace{.7mm}\solidrule[.3mm]\hspace{.7mm}\solidrule[.3mm]\hspace{.7mm}\solidrule[.3mm]\hspace{.7mm}\solidrule[.3mm]\hspace{.7mm}}}
\newcommand\dashedrule{\mbox{%
\solidrule[1mm]\hspace{1mm}\solidrule[1mm]\hspace{1mm}\solidrule[1mm]\hspace{1mm}\solidrule[1mm]\hspace{1mm}\solidrule[1mm]\hspace{1mm}}}
\newcommand\dasheddotted{\mbox{%
\solidrule[1mm]\hspace{.4mm}\solidrule[.2mm]\hspace{.4mm}\solidrule[1mm]\hspace{.4mm}\solidrule[.2mm]\hspace{.4mm}%
\solidrule[1mm]\hspace{.4mm}\solidrule[.2mm]\hspace{.4mm}\solidrule[1mm]\hspace{.4mm}\solidrule[.2mm]\hspace{.4mm}%
\solidrule[1mm]\hspace{.4mm}\solidrule[.2mm]\hspace{.4mm}}}

\director{Walter W. Piegorsch \& Yves A. Lussier}
\author{A. Grant Schissler}
\title{Statistical and Bioinformatic Contributions to Single-Subject Transcriptome Analytics}
\date{2016}
    \makeindex

%\ifpdf \pdfinfo{ /Author (Qijun Fang) /Title  (Hierarchical Bayesian Benchmark Dose Analysis) } \fi


\begin{document}
\maketitle

\chapter*{Acknowledgments}
First and foremost, thanks go to my advisor Dr. Walter Piegorsch. For what I have learned in wide areas of Statistics through him, his excellent guidance in this project as well as his constant encouragement and support throughout the years. His creative ideas and the time we spent on discussion were absolutely essential to the completion of this project.
%
%Thanks are due my dissertation committee members, Drs. Katherine Y. Barnes, Rabi Bhattacharya, Anton Westveld, D. Dean Billheimer and Chengcheng Hu for their precious suggestions during my research on this project. Special thanks are due Dr. Rabi Bhattacharya for giving me wonderful lectures in advanced theoretic statistics and offering me many other great helps during my doctoral study. Also, thanks are due my colleagues in the University of North Carolina at Wilmington, Drs. Susan J.~Simmons, Cuixian (Tracy) Chen and Yishi Wang for their helpful comments and suggestions during the preparation of the publications regarding to this research. I also want to thank all who have taught me mathematics and statistics through those interesting courses.
%
%Further thanks are due my supervisor for my internship at the Biostatistis and Data Management department (BADM) of the Ventana Medical System, Inc, Dr. James Ranger-Moore for providing me funding for my doctoral study, broadening my view of statistical applications in medical industry and allowing me to use the BADM workstation for computational intensive tasks in this project. Special thanks are due my manager Dr. Isaac Bai for teaching me to use SAS software and sharing with me his experience working in industry. I would also like to thank all my other colleagues in Ventana for it is so pleasant to work with them during the past five years.
%

Finally I would like to thank my wife Fang for her love and unconditional support.

\chapter*{Dedication}
\thispagestyle{topright}
\begin{center}To my wife, my parents, my grandparents and my arriving baby.\end{center}

%\begin{vim_bug_workaround}
%\end{vim_bug_workaround}

\tableofcontents

\listoffigures
\listoftables

\chapter*{Abstract}
\noindent This dissertation serves as a unifying document of three related articles developed during my dissertation research. The projects described involve the development of single-subject transcriptome (i.e. gene expression data) methodology for precision medicine and related applications. While working in this setting, traditional statistical approaches are largely unavailable due to prohibitive sample size and lack of independent replication. This leads one to rely on informatic devices including knowledge base integration (e.g., gene set annotations; Section \ref{sec:nof1pathways}) and external data sources (e.g., estimation of inter-gene correlation; Appendix \ref{App:C}). Common statistical themes include multivariate statistics (such as Mahalanobis distance and copulas) and computational-extensive significance testing. Briefly, the first work\footnote{Appendix \ref{App:A}: Dynamic changes of RNA-sequencing expression for precision medicine: N-of-1-\emph{pathways} Mahalanobis distance within pathways of single subjects predicts breast cancer survival} describes the development of clinically relevant single-subject metrics of gene set (pathway) differential expression, N-of-1-\emph{pathways} Mahalanobis distance (MD) scores. Next, the second article\footnote{Appendix \ref{App:C}: Testing for differentially expressed genetic pathways with single-subject N-of-1 data in the presence of inter-gene correlation} describes a method which overcomes a major shortcoming of the MD framework by accounting for inter-gene correlation. Lastly, the statistics developed in the previous works are re-purposed to analyze single-cell RNA-sequencing data derived from rare cells\footnote{Appendix \ref{App:E}: Analysis of aggregated cell-cell statistical distances within pathways unveils therapeutic-resistance mechanisms in circulating tumor cells}. Importantly, these works represent an interdisciplinary effort and I hope to show that creative solutions for pressing issues become possible at the intersection of statistics, biology, medicine, and computer science.

\vspace{1pc}

\noindent Statistical Advisor: Walter W. Piegorsch\\
Biomedical Informatic Advisor: Yves A. Lussier

\chapter{Introduction}\label{Chap:Intro}

\section{Article-based dissertation and organization}\label{sec:org}

format of this document
unaltered published format

\section{Single-subject transcriptome analytics}\label{sec:nof1}

\indent \indent The bulk of the projects described below involve the development of single-subject transcriptome (i.e. gene expression data) methodology for precision medicine. The motivation for single-subject biomedical methods is straightforward. Biology is exceeding complex - each patient is inherently unique. Demographics, genetics, and epi-genetics (to name a few) all contribute to a patient's diagnosis, treatment, and prognosis. Often a patient is given a drug that has, on average, a high success rate for their medical condition, but inexplicably the patient fails to respond and suffers a poor outcome. What if we can make treatment decisions and predictions for that \emph{individual's} dysregulated cellular mechanisms? 

The applied problem of single-subject studies is intriguing, but as statisticians we are trained to make inferences on population parameters, based on (often independently) sampled observations. Numerous philosophical and technical issues arise when dealing with only a single patient's data. (It is worth noting that Fisher's famous tea-tasting experiment can be viewed as a \emph{single-subject} study.) That being said, applications for statistical \emph{single-subject inference} have created a demand for developing this unexplored area. From a statistical research standpoint, \emph{single-subject inference} poses many open problems that involve statistical topics including multivariate statistics, computing, machine learning, high-dimensional data analysis, small sample analysis, paired-sample statistics, big data, and data visualization.

\section{N-of-1-\emph{pathways} framework}\label{sec:nof1pathways}

The first project of my dissertation research concerned the ``statistical component'' of the single-subject framework, N-of-1-\emph{pathways} developed in the lab which I work (\cite{Gardeux2014}). The basic goal is to discover single-subject dysregulated genetic pathways from gene expression data. Two paired samples are obtained from a patient, one from a baseline and one from a case condition (e.g., blood before and after treatment, tumor and normal tissue, etc.). The messenger-RNA (mRNA) expression is then quantified from the samples (via RNA-sequencing or microarrays). Next the workflow takes an informatic turn - the mRNAs are annotated to gene sets (pathways) that are functionally related via knowledgebases, such as Gene Ontology (\cite{Ashburner2000}). This is in effect a dimension-reduction technique that also provides mechanistic interpretation (indeed single biomarkers, mRNA or otherwise, have been notoriously difficult to use in creating therapies or classifiers). The last step in the framework is the statistical component. The differential expression within each pathway is quantified and a test of hypotheses is conducted. We coined the phrase \emph{differentially expressed pathway} (DEP) to describe when a null hypothesis of equal expression is rejected (as an analog to the well-known DEG; differentially expressed gene). 


\chapter{Developing N-of-1-\emph{pathways} \\Mahalanobis distance} \label{Chap:md}

\indent \indent In the foundational paper by \citet{Gardeux2014}, the statistical component of N-of-1-\emph{pathways} employed a Wilcoxon signed-rank test, yielding a P-value and its associated signed z-score (as a proxy for effect size). As an alternative, \citet{Schissler2015} introduced the Mahalanobis distance (MD) within pathways to quantify differential pathway expression. Informally speaking, applied to the DEP setting this distance is a covariance-adjusted average, signed (log) fold change of expression from baseline to case conditions. The adjustment accounts for the variance-covariance structure observed between the paired expression values within the gene set - mRNA fold changes in a tightly correlated pathway are more heavily weighted. The average MD pathway score was seen to be interpretable on the scale of measurement (fold change of expression) than the Wilcoxon test statistic or z-score transformation of its P-value. As such, the MD statistical component of N-of-1-\emph{pathways} provided a meaningful improvement in quantifying effect. This is a critical advancement in the field of single-subject transcriptome analytics. Moreover, these single-subject metrics were seen to be predictive of breast cancer survival (a first in this context). Appendix \ref{App:A} reproduces the complete \citet{Schissler2015} article and Appendix \ref{App:B} contains the corresponding supplementary material.

The development of MD was a joint work with an interdisciplinary team. Drs. YA Lussier, V Gardeux, and I Achour supplied the informatic methodology and biomedical phenotypes. Myself and Dr. WW Piegorsch developed the MD metric and bootstrapping procedure to assess significant differential pathway expression. I performed the analyses, designed and executed the simulations, and wrote the first draft of the article. Of particular note, I conceived the experiments whose results are displayed in Appendix \ref{App:A}, Table 3. Briefly, the sample pairings were changed to compare single samples across patients (as opposed to case-vs.-baseline within patient). This experiment was a precursor for the third work contained in this dissertation (see Chapter \ref{Chap:ctcs} and Appendix \ref{App:E}). 

\chapter{Accounting for inter-gene correlation in N-of-1-\emph{pathways}} \label{Chap:ct}

The MD formulation (Section \ref{sec:md}; \cite{Schissler2015}) described above appears to hold some promise. However, only half of the initial goal was achieved. The other major shortcoming of the Wilcoxon approach, besides eliminating the loss of information from ranking, is the assumption of inter-gene independence in the calculation of the P-value. Genes are known to be co-expressed, especially within a curated gene set (\cite{Tamayo2016}) and, moreover, when measurements are obtained from a single subject. Correlation in the observations often leads to poor standard error (SE) estimates and positive correlation may drastically inflate false positive errors (due to underestimation of the SE). This observation leads naturally to study the impact of inter-gene correlation in the N-of-1-\emph{pathways} framework.

I recently submitted a paper describing an improvement of the statistical component of N-of-1-\emph{pathways}. This work is geared towards a statistical audience. Importantly, the article introduces a method that accounts for inter-gene correlation and yields satisfactory false positive rates with non-trivial co-expression in the pathway. In my previous works, I typically simulated mRNA expression via independent negative binomial or Poisson modeling assumptions. Here, I used copulas (\cite{Genest2007,Yan2007}) to create multivariate distributions that simulate more authentic pathway expression. Preliminary results are promising and represent a significant advance in the statistical theory of \emph{single-subject inference} in the context gene expression analytics.

- refer to Appendix A \\
- critical picture\\
- my contribution to the project, mention last table as precursor to CTCs\\ 
- contribution to the field\\
- Connect to MD
- footnote where it is submitted
\chapter{Applying N-of-1-\emph{pathways} MD to\\analyze single-cell RNA-sequencing} \label{Chap:ctcs}

\indent \indent With an aim to discover what other insights the paired statistics resulting from N-of-1-\emph{pathways} MD could obtain, we modified the framework to analyze single-cell RNA-sequencing (scRNA-seq) data. Each cell in an organism is unique and aggregating single-cell signal across many cells dilutes and muddies the picture obtained from scRNA-seq. The development of paired, mechanism-anchored statistics between cells lends insight into cell-cell heterogeneity that is obfuscated by other methods and mitigates sample size requirements of cohort-based statistics. I developed a new framework to analyze single-cell transcriptomes from circulating tumor cells (CTCs) isolated from blood samples obtained from prostate cancer patients (Appendix \ref{App:E}; \cite{Patel2014,Schissler2016}). By aggregating the cell-cell MD pathway scores, therapeutic-resistance mechanisms were identified from only thirteen patients (five drug-resistant and eight drug-na\"{\i}ve). Further, novel visualizations of cell-cell pathway heterogeneity were developed by modifying wind rose plots (\cite{Court1963}). This project illustrated the flexibility of these methods outside the scope of \emph{single-subject inference} to the arena of small sample cohort sizes, enabling the study of rare diseases and smaller scale, limited-resource projects.

Again, the development of this article was a joint work. I proposed the study and, along with Dr. YA Lussier, conceived the notion of computing pathway scores for every pair of cells. I developed the three perspectives obtainable from different aggregation approaches. Myself and Dr. WW Piegorsch developed the ``clustered'' bootstrap to account for correlation between cells derived from the same patient. Dr. YA and I designed the rose plots. Finally, I authored the first draft.

\chapter{\large Concluding remarks}\label{Chap:conclusion}
\indent\indent The final chapter of this dissertation serves as a summary of the work presented in the previous chapters.

% 
%In the first chapter of the dissertation, an overview of the process of quantitative risk assessment and the steps it entails was given. The dose-response modeling and quantitative risk estimation was addressed.  Here, risk $R(d)$ was defined as the probability of a subject exhibiting a pre-defined adverse effect when exposed to a particular dose level, $d$, of a hazardous agent. Two key assumptions were made on this fundamental definition of risk.  The first assumption was that risk is a monotone increasing function. The second assumption was that non-zero background risk at $d=0$ may exist in the population of interest. To correct for this spontaneous background risk, excess risks such as additional risk and extra risk were further introduced. These excess risks are more commonly used in quantitative risk estimation. After that, difficulties in low-dose risk estimation were discussed and a modern technique---benchmark dose analysis---was introduced. In benchmark dose analysis, one uses the functional specification for $R(d)$ to provide low-dose estimates for risk and/or excess risk. To estimate the benchmark dose (BMD) and the corresponding benchmark dose lower limit (BMDL) in quantal-response data setting (as is the focus of this dissertation), the traditional frequentist method through parametric maximum likelihood estimation was first introduced, and then a thorough literature review on established parametric hierarchical Bayesian benchmark dose analysis was provided.
%
%Chapter \ref{Chap:Repar} introduced 8 popular quantal-response models as seen in the U.S. EPA's BMDS software \citep{davi12}. These were forms of generalized linear models expressed via traditional $\beta-$type parameters. Because these parameters usually don't have pertinent risk analytic interpretations, it is difficult to construct prior distributions for them. As a result, objective, even improper, priors had to be used instead. In order to embrace useful prior information into the Bayesian hierarchical modeling, reparameterizations for these 8 models were performed by using the target quantity, BMD ($\xi$), background risk ($\gamma_0$) and in three-parameter models, also a risk at some dose level (usually the risk at the highest level, $\gamma_1$) as the new parameters. Mathematical forms have been derived; these reparameterizations present more burdensome notation for $R(d)$, however, allow benchmark analysts to formulate a clearer and more application-oriented hierarchical model, from which to produce inferences on BMD.
%
%The hierarchical Bayesian modeling framework was introduced in Chapter \ref{Chap:Bayesian Frame}. This included three major topics: prior specification, posterior approximation and Bayesian estimation. For positive quantity $\xi$, inverse gamma prior was specified by default; for probability quantities $\gamma_0$ and $\gamma_1$ (also for the risks at any other dose level), Beta prior was specified. Prior elicitation process using first quartile and median was introduced. When elicitation is not available, proper objective priors using $\xi\sim IG(0.001, 0.001)$ and $\gamma_0, \gamma_1 \sim Beta(\frac12, \frac12)$ were employed and recommended. Joint posterior density was approximated by using an adaptive Metropolis algorithm introduced in \citet{anth08}, convergence diagnosis and burn-in determination were applied mimicking a method introduced in \citet{gewe92}. Several decision theoretical Bayesian estimators were discussed and the posterior lower tercile estimator resulted from bi-linear asymmetric loss function was recommended for BMD point estimation. With the AM sample, 95\% BMDL was easily estimated as the 5th lower percentile of the AM sample. To investigate and assess the prior sensitivity on $\xi$, an $\epsilon-$contamination study followed by \citet[\S7.15]{ohag94} was introduced.
%
%The increasing number of available models for benchmark dose analysis brings in the issue of model uncertainty and adequacy. Chapter \ref{Chap:BMA} discussed this important issue. Due to the difficulty of finding reliable selecting criterion for model selection, a Bayesian model averaging (BMA) method introduced by \citet{homa99} was employed. In this method, those 8 quantal-response models were chosen as the uncertainty class and a mixture distribution for $\xi$ was constructed using the posterior model probabilities as the weights. The BMA BMD is not pursued while the BMA BMDL which satisfies the corresponding probability statement is approximated by using a `direct method' from each model-specific AM sample.
%
%In Chapter \ref{Chap:Example}, application of the proposed hierarchical Bayesian method was illustrated via a real carcinogenicity study performed by U.S. National Toxicology Program (NTP) on the chemical cumene. BMD and BMDL were first estimated by using the popular quantal linear model. To investigate prior sensitivity, three $\epsilon-$contaminating scenarios were considered, and it was shown that the estimates from quantal linear model was reasonably robust. After that, another popular model---logistic model---was fitted to the data and its estimates was found to be significantly different from quantal linear model's. Facing the issue of model uncertainty, the remaining 6 models were then fitted to the data and the Bayesian model averaged BMDL was produced. This showed an example that BMA adjustment could free the risk assessors from model inadequacies and inferential uncertainties which were frequently encountered when committing to a single-model benchmark analysis.
%
%Chapter \ref{Chap:Simu} aimed to evaluate the performance of the proposed method via a simulation study. 8 reparameterized quantal response models were fitted to a set of 2000 data sets generated by each of these models under 5 configurations and 3 sample-sizes. Asymptotic normality of the point estimates of BMD appeared to be satisfied when fitting the correct model to the data. The model-specific BMDL also appeared to converged to nominal as $N$ increased when the correct model was fitted to the data. When model were misspecified, however, both the point estimates and the BMDL became unreliable. The BMA BMDL, on the other hand, could provide much more stable and acceptable estimates.
%
%With the conclusion of this portion of the research, there are several natural extensions for further study. Mimicking previous Bayesian benchmark analysis methods \citep{shsm11,shsm12,shao12}, the current method assumed that the unknown parameters enter the joint prior independently. The correlation between parameters may not be ignored, however. If correlation between parameters can be elicited and successfully incorporated into the prior hierarchy, the BMD/BMDL estimation may be further improved. In addition, objective prior specification can be approached via many strategies when elicitation breaks down. The current method assumed independent $IG(0.001, 0.001)$ and $Beta(\frac12,\frac12)$ priors. Other forms may be pertinent, however, and these choices might still bring possible subjectivity to the proposed hierarchical model. It is also of interest to investigate how the proposed approach operates under different design configurations. The current research focused on a geometric, four-dose design, arguably the quintessential standard in cancer and laboratory-animal toxicology testing. Greater information might be gained about the pattern of dose response, however, and therefore about the BMD, if the number of doses is increased and/or the dose spacings are changed. Experimental design for dose-response studies with focus on the BMD is an emerging area in the statistical literature \citep{muri09,ober10,sand08,shsm12} and how to optimally design/allocate experimental resources for BMD estimation and inferences under a Bayesian paradigm is an open question.


% \input{chapter6}
%\input{chapter7}
%

\setboolean{@twoside}{false}

\printbibliography[heading=refs]
\appendix
\renewcommand{\thetable}{A.\arabic{table}}
\chapter{Dynamic changes of RNA-sequencing expression for precision medicine: N-of-1-\emph{pathways} Mahalanobis distance within pathways of single subjects predicts breast cancer survival}

\includepdf[pages=-]{md.pdf}
\renewcommand{\thetable}{B.\arabic{table}}
\chapter{Coverage rates for model specific BMDL and BMA BMDL}

\indent\indent Chapter \ref{Chap:Simu} illustrated the coverage rates for model specific BMDL and BMA BMDL via spaghetti plots (See Figures \ref{fig:SpaghettiM12M4}, \ref{fig:SpaghettiM52M8} and \ref{fig:BMABMDLCoverage}). Here, the actual values of coverage rates are provided supplement to those plots. Model codes are from Table \ref{dose_response}.

% latex table generated in R 3.0.2 by xtable 1.7-1 package
% Wed Dec 25 16:33:36 2013
\begin{table}[ht]
\caption{Coverage rates when fitting logistic model (M1) model to the simulated data generated by 8 models, under 5 configurations and 3 sample-sizes.}\label{cover1}
\renewcommand{\arraystretch}{1.25} %widens row widths
\vspace{8pt}
\centering
\scalebox{0.84}{
\begin{tabular}{lrrrrrrrrr}
  \hline
Configuration & $N$ & M1 & M2 & M3 & M4 & M5 & M6 & M7 & M8 \\
  \hline
C1 & 25 & 0.715 & 0.714 & 0.732 & 0.726 & 0.723 & 0.722 & 0.726 & 0.726 \\
& 50 & 0.791 & 0.797 & 0.807 & 0.801 & 0.798 & 0.810 & 0.803 & 0.801 \\
& 1000 & 0.937 & 0.953 & 0.989 & 0.969 & 0.973 & 0.979 & 0.987 & 0.978 \\
  \hline
  C2 & 25 & 0.773 & 0.753 & 0.482 & 0.868 & 0.820 & 0.817 & 0.817 & 0.820 \\
& 50 & 0.834 & 0.802 & 0.413 & 0.920 & 0.881 & 0.875 & 0.871 & 0.876 \\
& 1000 & 0.936 & 0.805 & 0.000 & 1.000 & 0.998 & 0.996 & 0.985 & 0.997 \\
  \hline
  C3 & 25 & 0.898 & 0.803 & 0.002 & 0.909 & 0.004 & 0.052 & 0.136 & 0.012 \\
& 50 & 0.917 & 0.770 & 0.000 & 0.929 & 0.000 & 0.002 & 0.020 & 0.000 \\
& 1000 & 0.954 & 0.065 & 0.000 & 0.979 & 0.000 & 0.000 & 0.000 & 0.000 \\
  \hline
  C4 & 25 & 0.877 & 0.873 & 0.203 & 0.999 & 0.936 & 0.990 & 0.998 & 0.976 \\
& 50 & 0.906 & 0.899 & 0.056 & 1.000 & 0.975 & 1.000 & 1.000 & 0.996 \\
& 1000 & 0.942 & 0.895 & 0.000 & 1.000 & 1.000 & 1.000 & 1.000 & 1.000 \\
  \hline
  C5 & 25 & 0.942 & 0.876 & 0.000 & 0.996 & 0.989 & 1.000 & 1.000 & 0.994 \\
& 50 & 0.940 & 0.841 & 0.000 & 0.999 & 0.994 & 1.000 & 1.000 & 0.998 \\
& 1000 & 0.952 & 0.213 & 0.000 & 1.000 & 1.000 & 1.000 & 1.000 & 1.000 \\
   \hline
\end{tabular}
}
\end{table}

% latex table generated in R 3.0.2 by xtable 1.7-1 package
% Wed Dec 25 16:33:36 2013
\begin{table}[ht]
\caption{Coverage rates when fitting probit model (M2) model to the simulated data generated by 8 models, under 5 configurations and 3 sample-sizes.}\label{cover1}
\renewcommand{\arraystretch}{1.25} %widens row widths
\vspace{8pt}
\centering
\scalebox{0.84}{
\begin{tabular}{lrrrrrrrrr}
  \hline
Configuration & $N$ & M1 & M2 & M3 & M4 & M5 & M6 & M7 & M8 \\
  \hline
C1 & 25 & 0.733 & 0.731 & 0.750 & 0.740 & 0.736 & 0.746 & 0.746 & 0.746 \\
& 50 & 0.799 & 0.803 & 0.815 & 0.811 & 0.811 & 0.812 & 0.815 & 0.815 \\
& 1000 & 0.915 & 0.936 & 0.988 & 0.953 & 0.962 & 0.973 & 0.981 & 0.972 \\
  \hline
  C2 & 25 & 0.815 & 0.796 & 0.566 & 0.897 & 0.847 & 0.846 & 0.849 & 0.851 \\
& 50 & 0.866 & 0.843 & 0.530 & 0.933 & 0.899 & 0.896 & 0.895 & 0.895 \\
& 1000 & 0.979 & 0.940 & 0.000 & 1.000 & 1.000 & 0.998 & 0.996 & 0.999 \\
  \hline
  C3 & 25 & 0.951 & 0.880 & 0.004 & 0.967 & 0.014 & 0.103 & 0.226 & 0.022 \\
& 50 & 0.979 & 0.908 & 0.000 & 0.983 & 0.000 & 0.011 & 0.065 & 0.000 \\
& 1000 & 1.000 & 0.952 & 0.000 & 1.000 & 0.000 & 0.000 & 0.000 & 0.000 \\
  \hline
  C4 & 25 & 0.885 & 0.880 & 0.164 & 0.998 & 0.942 & 0.991 & 0.997 & 0.979 \\
& 50 & 0.916 & 0.906 & 0.034 & 1.000 & 0.979 & 1.000 & 1.000 & 0.997 \\
& 1000 & 0.968 & 0.940 & 0.000 & 1.000 & 1.000 & 1.000 & 1.000 & 1.000 \\
  \hline
  C5 & 25 & 0.975 & 0.930 & 0.000 & 0.999 & 0.995 & 1.000 & 1.000 & 0.998 \\
& 50 & 0.983 & 0.940 & 0.000 & 1.000 & 0.999 & 1.000 & 1.000 & 0.999 \\
& 1000 & 1.000 & 0.942 & 0.000 & 1.000 & 1.000 & 1.000 & 1.000 & 1.000 \\
   \hline
\end{tabular}
}
\end{table}

% latex table generated in R 3.0.2 by xtable 1.7-1 package
% Wed Dec 25 16:33:36 2013
\begin{table}[ht]
\caption{Coverage rates when fitting quantal linear model (M3) model to the simulated data generated by 8 models, under 5 configurations and 3 sample-sizes.}\label{cover1}
\renewcommand{\arraystretch}{1.25} %widens row widths
\vspace{8pt}
\centering
\scalebox{0.84}{
\begin{tabular}{lrrrrrrrrr}
  \hline
Configuration & $N$ & M1 & M2 & M3 & M4 & M5 & M6 & M7 & M8 \\
  \hline
C1 & 25 & 0.724 & 0.736 & 0.827 & 0.723 & 0.769 & 0.764 & 0.749 & 0.764 \\
& 50 & 0.711 & 0.734 & 0.854 & 0.724 & 0.766 & 0.772 & 0.762 & 0.765 \\
& 1000 & 0.102 & 0.197 & 0.943 & 0.114 & 0.365 & 0.361 & 0.335 & 0.359 \\
  \hline
  C2 & 25 & 0.923 & 0.915 & 0.858 & 0.952 & 0.942 & 0.939 & 0.940 & 0.939 \\
& 50 & 0.940 & 0.941 & 0.877 & 0.966 & 0.956 & 0.954 & 0.952 & 0.955 \\
& 1000 & 1.000 & 1.000 & 0.946 & 1.000 & 1.000 & 1.000 & 1.000 & 1.000 \\
  \hline
  C3 & 25 & 0.999 & 0.998 & 0.911 & 0.999 & 0.940 & 0.977 & 0.989 & 0.953 \\
& 50 & 1.000 & 1.000 & 0.924 & 1.000 & 0.961 & 0.991 & 0.996 & 0.973 \\
& 1000 & 1.000 & 1.000 & 0.951 & 1.000 & 0.998 & 1.000 & 1.000 & 1.000 \\
  \hline
  C4 & 25 & 0.983 & 0.983 & 0.901 & 0.999 & 0.989 & 0.998 & 0.999 & 0.995 \\
& 50 & 0.999 & 0.999 & 0.923 & 1.000 & 1.000 & 1.000 & 1.000 & 1.000 \\
& 1000 & 1.000 & 1.000 & 0.946 & 1.000 & 1.000 & 1.000 & 1.000 & 1.000 \\
  \hline
  C5 & 25 & 1.000 & 1.000 & 0.929 & 1.000 & 1.000 & 1.000 & 1.000 & 1.000 \\
& 50 & 1.000 & 1.000 & 0.941 & 1.000 & 1.000 & 1.000 & 1.000 & 1.000 \\
& 1000 & 1.000 & 1.000 & 0.948 & 1.000 & 1.000 & 1.000 & 1.000 & 1.000 \\
   \hline
\end{tabular}
}
\end{table}
% latex table generated in R 3.0.2 by xtable 1.7-1 package
% Wed Dec 25 16:33:36 2013
\begin{table}[ht]
\caption{Coverage rates when fitting quantal quadratic model (M4) model to the simulated data generated by 8 models, under 5 configurations and 3 sample-sizes.}\label{cover1}
\renewcommand{\arraystretch}{1.25} %widens row widths
\vspace{8pt}
\centering
\scalebox{0.84}{
\begin{tabular}{lrrrrrrrrr}
  \hline
Configuration & $N$ & M1 & M2 & M3 & M4 & M5 & M6 & M7 & M8 \\
  \hline
C1 & 25 & 0.765 & 0.768 & 0.772 & 0.781 & 0.776 & 0.782 & 0.783 & 0.783 \\
& 50 & 0.825 & 0.827 & 0.828 & 0.831 & 0.831 & 0.840 & 0.839 & 0.837 \\
& 1000 & 0.897 & 0.936 & 0.991 & 0.944 & 0.969 & 0.978 & 0.984 & 0.975 \\
  \hline
  C2 & 25 & 0.656 & 0.607 & 0.263 & 0.809 & 0.726 & 0.723 & 0.705 & 0.726 \\
& 50 & 0.663 & 0.595 & 0.084 & 0.861 & 0.770 & 0.763 & 0.757 & 0.768 \\
& 1000 & 0.009 & 0.000 & 0.000 & 0.940 & 0.258 & 0.207 & 0.167 & 0.235 \\
  \hline
  C3 & 25 & 0.897 & 0.731 & 0.000 & 0.919 & 0.000 & 0.000 & 0.002 & 0.000 \\
& 50 & 0.903 & 0.625 & 0.000 & 0.933 & 0.000 & 0.000 & 0.000 & 0.000 \\
& 1000 & 0.760 & 0.000 & 0.000 & 0.958 & 0.000 & 0.000 & 0.000 & 0.000 \\
  \hline
  C4 & 25 & 0.003 & 0.002 & 0.000 & 0.912 & 0.018 & 0.434 & 0.621 & 0.109 \\
& 50 & 0.000 & 0.000 & 0.000 & 0.931 & 0.000 & 0.227 & 0.465 & 0.010 \\
& 1000 & 0.000 & 0.000 & 0.000 & 0.945 & 0.000 & 0.000 & 0.000 & 0.000 \\
  \hline
  C5 & 25 & 0.274 & 0.102 & 0.000 & 0.942 & 0.679 & 1.000 & 1.000 & 0.825 \\
& 50 & 0.070 & 0.004 & 0.000 & 0.948 & 0.523 & 1.000 & 1.000 & 0.767 \\
& 1000 & 0.000 & 0.000 & 0.000 & 0.952 & 0.000 & 1.000 & 1.000 & 0.019 \\
   \hline
\end{tabular}
}
\end{table}
% latex table generated in R 3.0.2 by xtable 1.7-1 package
% Wed Dec 25 16:33:36 2013
\begin{table}[ht]
\caption{Coverage rates when fitting two-stage model (M5) model to the simulated data generated by 8 models, under 5 configurations and 3 sample-sizes.}\label{cover1}
\renewcommand{\arraystretch}{1.25} %widens row widths
\vspace{8pt}
\centering
\scalebox{0.84}{
\begin{tabular}{lrrrrrrrrr}
  \hline
Configuration & $N$ & M1 & M2 & M3 & M4 & M5 & M6 & M7 & M8 \\
  \hline
C1 & 25 & 0.914 & 0.918 & 0.941 & 0.918 & 0.928 & 0.927 & 0.925 & 0.927 \\
& 50 & 0.890 & 0.898 & 0.954 & 0.894 & 0.917 & 0.919 & 0.923 & 0.916 \\
& 1000 & 0.839 & 0.883 & 0.986 & 0.904 & 0.930 & 0.942 & 0.959 & 0.938 \\
  \hline
  C2 & 25 & 0.932 & 0.928 & 0.950 & 0.924 & 0.931 & 0.933 & 0.942 & 0.927 \\
& 50 & 0.949 & 0.950 & 0.953 & 0.945 & 0.947 & 0.952 & 0.945 & 0.951 \\
& 1000 & 0.987 & 0.980 & 0.794 & 1.000 & 0.991 & 0.987 & 0.977 & 0.988 \\
  \hline
  C3 & 25 & 1.000 & 0.999 & 0.738 & 1.000 & 0.819 & 0.939 & 0.979 & 0.857 \\
& 50 & 1.000 & 1.000 & 0.702 & 1.000 & 0.797 & 0.938 & 0.981 & 0.842 \\
& 1000 & 1.000 & 0.974 & 0.721 & 1.000 & 0.912 & 0.993 & 1.000 & 0.962 \\
  \hline
  C4 & 25 & 0.983 & 0.982 & 0.725 & 0.999 & 0.994 & 1.000 & 1.000 & 0.999 \\
& 50 & 0.982 & 0.980 & 0.662 & 1.000 & 0.995 & 1.000 & 1.000 & 1.000 \\
& 1000 & 0.948 & 0.941 & 0.683 & 1.000 & 0.952 & 1.000 & 1.000 & 0.988 \\
  \hline
  C5 & 25 & 1.000 & 1.000 & 0.616 & 1.000 & 1.000 & 1.000 & 1.000 & 1.000 \\
& 50 & 1.000 & 0.999 & 0.634 & 1.000 & 1.000 & 1.000 & 1.000 & 1.000 \\
& 1000 & 0.818 & 0.684 & 0.722 & 1.000 & 0.990 & 1.000 & 1.000 & 0.999 \\
   \hline
\end{tabular}
}
\end{table}
% latex table generated in R 3.0.2 by xtable 1.7-1 package
% Wed Dec 25 16:33:36 2013
\begin{table}[ht]
\caption{Coverage rates when fitting log-logistic model (M6) model to the simulated data generated by 8 models, under 5 configurations and 3 sample-sizes.}\label{cover1}
\renewcommand{\arraystretch}{1.25} %widens row widths
\vspace{8pt}
\centering
\scalebox{0.84}{
\begin{tabular}{lrrrrrrrrr}
  \hline
Configuration & $N$ & M1 & M2 & M3 & M4 & M5 & M6 & M7 & M8 \\
  \hline
C1 & 25 & 0.738 & 0.732 & 0.801 & 0.751 & 0.754 & 0.756 & 0.759 & 0.757 \\
& 50 & 0.711 & 0.713 & 0.767 & 0.725 & 0.722 & 0.733 & 0.721 & 0.742 \\
& 1000 & 0.878 & 0.882 & 0.922 & 0.927 & 0.909 & 0.929 & 0.946 & 0.926 \\
  \hline
  C2 & 25 & 0.996 & 0.997 & 0.995 & 0.998 & 0.998 & 0.996 & 0.999 & 0.998 \\
& 50 & 0.995 & 0.992 & 0.989 & 0.994 & 0.995 & 0.992 & 0.994 & 0.995 \\
& 1000 & 0.975 & 0.975 & 0.970 & 0.960 & 0.971 & 0.963 & 0.940 & 0.966 \\
  \hline
  C3 & 25 & 0.969 & 0.966 & 0.973 & 0.973 & 0.969 & 0.973 & 0.981 & 0.967 \\
& 50 & 0.947 & 0.948 & 0.958 & 0.954 & 0.955 & 0.966 & 0.977 & 0.957 \\
& 1000 & 0.905 & 0.879 & 0.870 & 0.931 & 0.835 & 0.954 & 0.980 & 0.878 \\
  \hline
  C4 & 25 & 0.964 & 0.961 & 0.968 & 0.980 & 0.965 & 0.984 & 0.985 & 0.971 \\
& 50 & 0.930 & 0.930 & 0.954 & 0.951 & 0.930 & 0.973 & 0.979 & 0.951 \\
& 1000 & 0.545 & 0.507 & 0.720 & 0.825 & 0.553 & 0.948 & 0.986 & 0.785 \\
  \hline
  C5 & 25 & 0.785 & 0.754 & 0.901 & 0.850 & 0.829 & 0.951 & 0.961 & 0.845 \\
& 50 & 0.639 & 0.607 & 0.814 & 0.788 & 0.749 & 0.956 & 0.964 & 0.786 \\
& 1000 & 0.000 & 0.000 & 0.000 & 0.051 & 0.008 & 0.962 & 0.957 & 0.042 \\
   \hline
\end{tabular}
}
\end{table}

% latex table generated in R 3.0.2 by xtable 1.7-1 package
% Wed Dec 25 16:33:36 2013
\begin{table}[ht]
\caption{Coverage rates when fitting log-probit model (M7) model to the simulated data generated by 8 models, under 5 configurations and 3 sample-sizes.}\label{cover1}
\renewcommand{\arraystretch}{1.25} %widens row widths
\vspace{8pt}
\centering
\scalebox{0.84}{
\begin{tabular}{lrrrrrrrrr}
  \hline
Configuration & $N$ & M1 & M2 & M3 & M4 & M5 & M6 & M7 & M8 \\
  \hline
C1 & 25 & 0.735 & 0.739 & 0.814 & 0.738 & 0.756 & 0.764 & 0.771 & 0.765 \\
& 50 & 0.713 & 0.724 & 0.763 & 0.724 & 0.732 & 0.732 & 0.727 & 0.737 \\
& 1000 & 0.833 & 0.846 & 0.900 & 0.896 & 0.875 & 0.897 & 0.923 & 0.895 \\
  \hline
  C2 & 25 & 0.999 & 0.998 & 0.995 & 0.998 & 0.997 & 0.998 & 0.998 & 0.997 \\
& 50 & 0.994 & 0.992 & 0.989 & 0.994 & 0.995 & 0.994 & 0.993 & 0.995 \\
& 1000 & 0.978 & 0.981 & 0.974 & 0.970 & 0.977 & 0.970 & 0.956 & 0.971 \\
  \hline
  C3 & 25 & 0.964 & 0.962 & 0.964 & 0.969 & 0.961 & 0.967 & 0.976 & 0.962 \\
& 50 & 0.940 & 0.936 & 0.949 & 0.948 & 0.942 & 0.960 & 0.969 & 0.942 \\
& 1000 & 0.884 & 0.872 & 0.718 & 0.914 & 0.683 & 0.902 & 0.954 & 0.739 \\
  \hline
  C4 & 25 & 0.957 & 0.957 & 0.965 & 0.973 & 0.959 & 0.981 & 0.983 & 0.968 \\
& 50 & 0.914 & 0.910 & 0.945 & 0.940 & 0.915 & 0.963 & 0.971 & 0.942 \\
& 1000 & 0.376 & 0.338 & 0.532 & 0.701 & 0.411 & 0.904 & 0.954 & 0.671 \\
  \hline
  C5 & 25 & 0.757 & 0.733 & 0.891 & 0.839 & 0.811 & 0.949 & 0.960 & 0.832 \\
& 50 & 0.606 & 0.570 & 0.788 & 0.769 & 0.723 & 0.949 & 0.963 & 0.768 \\
& 1000 & 0.000 & 0.000 & 0.000 & 0.046 & 0.004 & 0.954 & 0.948 & 0.022 \\
   \hline
\end{tabular}
}
\end{table}

% latex table generated in R 3.0.2 by xtable 1.7-1 package
% Wed Dec 25 16:33:36 2013
\begin{table}[ht]
\caption{Coverage rates when fitting Weibull model (M8) model to the simulated data generated by 8 models, under 5 configurations and 3 sample-sizes.}\label{cover1}
\renewcommand{\arraystretch}{1.25} %widens row widths
\vspace{8pt}
\centering
  \scalebox{0.84}{
\begin{tabular}{lrrrrrrrrr}
  \hline
Configuration & $N$ & M1 & M2 & M3 & M4 & M5 & M6 & M7 & M8 \\
  \hline
C1 & 25 & 0.949 & 0.960 & 0.972 & 0.965 & 0.967 & 0.968 & 0.965 & 0.969 \\
& 50 & 0.924 & 0.928 & 0.966 & 0.932 & 0.942 & 0.948 & 0.949 & 0.943 \\
& 1000 & 0.884 & 0.892 & 0.985 & 0.934 & 0.927 & 0.942 & 0.957 & 0.938 \\
  \hline
  C2 & 25 & 0.944 & 0.935 & 0.889 & 0.962 & 0.951 & 0.949 & 0.956 & 0.953 \\
& 50 & 0.956 & 0.941 & 0.858 & 0.973 & 0.966 & 0.965 & 0.959 & 0.965 \\
& 1000 & 0.973 & 0.967 & 0.803 & 0.954 & 0.967 & 0.957 & 0.930 & 0.962 \\
  \hline
  C3 & 25 & 0.963 & 0.953 & 0.649 & 0.966 & 0.720 & 0.860 & 0.914 & 0.764 \\
& 50 & 0.954 & 0.945 & 0.656 & 0.958 & 0.740 & 0.896 & 0.951 & 0.788 \\
& 1000 & 0.940 & 0.922 & 0.671 & 0.952 & 0.885 & 0.984 & 0.996 & 0.942 \\
  \hline
  C4 & 25 & 0.882 & 0.876 & 0.601 & 0.989 & 0.918 & 0.985 & 0.991 & 0.959 \\
& 50 & 0.891 & 0.882 & 0.597 & 0.972 & 0.920 & 0.982 & 0.990 & 0.957 \\
& 1000 & 0.868 & 0.843 & 0.598 & 0.956 & 0.862 & 0.994 & 0.999 & 0.953 \\
  \hline
  C5 & 25 & 0.899 & 0.880 & 0.528 & 0.950 & 0.938 & 0.994 & 0.993 & 0.948 \\
& 50 & 0.878 & 0.848 & 0.529 & 0.949 & 0.933 & 0.997 & 0.997 & 0.952 \\
& 1000 & 0.354 & 0.147 & 0.556 & 0.954 & 0.863 & 1.000 & 1.000 & 0.954 \\
   \hline
\end{tabular}
}
\end{table}

% latex table generated in R 3.0.2 by xtable 1.7-1 package
% Wed Dec 25 16:33:36 2013
\begin{table}[ht]
\caption{Coverage rates when applying Bayesian model averaging to the simulated data generated by 8 models, under 5 configurations and 3 sample-sizes.}\label{cover1}
\renewcommand{\arraystretch}{1.25} %widens row widths
\vspace{8pt}
\centering
  \scalebox{0.84}{
\begin{tabular}{lrrrrrrrrr}
  \hline
Configuration & $N$ & M1 & M2 & M3 & M4 & M5 & M6 & M7 & M8 \\
  \hline
C1 & 25 & 0.753 & 0.754 & 0.786 & 0.755 & 0.765 & 0.769 & 0.767 & 0.769 \\
& 50 & 0.804 & 0.808 & 0.844 & 0.817 & 0.820 & 0.823 & 0.827 & 0.825 \\
& 1000 & 0.909 & 0.924 & 0.967 & 0.948 & 0.936 & 0.952 & 0.964 & 0.950 \\
  \hline
  C2 & 25 & 0.901 & 0.895 & 0.838 & 0.933 & 0.916 & 0.914 & 0.917 & 0.917 \\
& 50 & 0.924 & 0.918 & 0.869 & 0.950 & 0.933 & 0.937 & 0.931 & 0.935 \\
& 1000 & 0.982 & 0.974 & 0.902 & 0.988 & 0.979 & 0.972 & 0.942 & 0.974 \\
  \hline
  C3 & 25 & 0.990 & 0.982 & 0.940 & 0.991 & 0.938 & 0.960 & 0.971 & 0.942 \\
& 50 & 0.989 & 0.973 & 0.923 & 0.993 & 0.927 & 0.955 & 0.970 & 0.935 \\
& 1000 & 0.989 & 0.895 & 0.951 & 0.992 & 0.984 & 0.985 & 0.990 & 0.984 \\
  \hline
  C4 & 25 & 0.976 & 0.975 & 0.930 & 0.998 & 0.983 & 0.997 & 1.000 & 0.989 \\
& 50 & 0.988 & 0.986 & 0.925 & 0.997 & 0.993 & 0.999 & 0.998 & 0.997 \\
& 1000 & 0.973 & 0.958 & 0.917 & 0.979 & 0.985 & 0.976 & 0.987 & 0.978 \\
  \hline
  C5 & 25 & 0.959 & 0.940 & 0.888 & 0.992 & 0.988 & 0.998 & 0.998 & 0.992 \\
& 50 & 0.960 & 0.910 & 0.900 & 0.994 & 0.992 & 1.000 & 0.999 & 0.995 \\
& 1000 & 0.974 & 0.876 & 0.922 & 0.988 & 0.974 & 0.968 & 0.957 & 0.956 \\
   \hline
\end{tabular}
}
\end{table} 
\includepdf[pages=-]{md_supplement.pdf}
\renewcommand{\thetable}{C.\arabic{table}}
\input{appendixC}
\includepdf[pages=-]{clustered_T_smmr_v10.pdf}
\renewcommand{\thetable}{D.\arabic{table}}
\chapter{Testing for differentially expressed genetic pathways with single-subject N-of-1 data in the presence of inter-gene correlation supplementary material}


\includepdf[pages=-]{clustered_T_suppl_v4.pdf}
\renewcommand{\thetable}{E.\arabic{table}}
\chapter{Analysis of aggregated cell-cell statistical distances within pathways unveils therapeutic-resistance mechanisms in circulating tumor cells}


\includepdf[pages=-]{ctcs.pdf}
\renewcommand{\thetable}{F.\arabic{table}}
% \chapter{Analysis of aggregated cell-cell statistical distances within pathways unveils therapeutic-resistance mechanisms in circulating tumor cells supplementary material}

\chapter{Supplementary material to Appendix E}


% \includepdf[pages=-, offset=0 -25]{ctcs_supplment.pdf}
\includepdf[pages=-]{ctcs_supplment.pdf}

%\clearpage
% \input{bibliography}
% \bibliography{/Users/grantschissler/Dropbox/bib/default, /Users/grantschissler/Dropbox/bib/library}

%\printbibliography


\end{document}
