\chapter{Developing N-of-1-\emph{pathways} \\Mahalanobis distance} \label{Chap:md}

\indent \indent In the foundational paper by \citet{Gardeux2014}, the statistical component of N-of-1-\emph{pathways} employed a Wilcoxon signed-rank test, yielding a P-value and its associated signed z-score (as a proxy for effect size). As an alternative, \citet{Schissler2015} introduced the Mahalanobis distance (MD) within pathways to quantify differential pathway expression. Informally speaking, applied to the DEP setting this distance is a covariance-adjusted average, signed (log) fold change of expression from baseline to case conditions. The adjustment accounts for the variance-covariance structure observed between the paired expression values within the gene set - mRNA fold changes in a tightly correlated pathway are more heavily weighted. The average MD pathway score was seen to be interpretable on the scale of measurement (fold change of expression) than the Wilcoxon test statistic or z-score transformation of its P-value. As such, the MD statistical component of N-of-1-\emph{pathways} provided a meaningful improvement in quantifying effect. This is a critical advancement in the field of single-subject transcriptome analytics. Moreover, these single-subject metrics were seen to be predictive of breast cancer survival (a first in this context). Appendix \ref{App:A} reproduces the complete \citet{Schissler2015} article and Appendix \ref{App:B} contains the corresponding supplementary material.
%
%The development of MD was a joint work with an interdisciplinary team. Drs. YA Lussier, V Gardeux, and I Achour supplied the informatic methodology and biomedical phenotypes. Myself and Dr. WW Piegorsch developed the MD metric and bootstrapping procedure to assess significant differential pathway expression. I performed the analyses, designed and executed the simulations, and wrote the first draft of the article. Of particular note, I conceived the experiments whose results are displayed in Appendix \ref{App:A}, Table 3. Briefly, the sample pairings were changed to compare single samples across patients (as opposed to case-vs.-baseline within patient). This experiment was a precursor for the third work contained in this dissertation (see Chapter \ref{Chap:ctcs} and Appendix \ref{App:E}). 
