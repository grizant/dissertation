\chapter{Developing N-of-1-\emph{pathways} \\Mahalanobis distance} \label{Chap:md}

\indent \indent In the foundational paper, the statistical component of N-of-1-\emph{pathways} employed a Wilcoxon signed-rank test, yielding a P-value and its associated signed z-score (as a proxy for effect size). As an alternative, we introduced the Mahalanobis distance (MD; Appendix \ref{App:A}; \cite{Schissler2015}) within pathways to quantify differential pathway expression. Informally speaking, this distance is a covariance-adjusted average, signed (log) fold change of expression from baseline to case conditions. The covariance-variance adjustment accounts for structure observed between the paired expression values within the gene set - mRNA fold changes in a tightly correlated pathway are more heavily weighted. The average MD pathway score was seen to be interpretable on the scale of measurement (fold change of expression) than the Wilcoxon test statistic or transformation of its P-value (z-score). As such, the MD statistical component of N-of-1-\emph{pathways} provided a meaningful improvement in quantifying effect. This is the first such advancement in the field single-subject transcriptome analytics to my knowledge. Moreover, these single-subject metrics were predictive of breast cancer survival (also a first in this context).

The development of MD was a joint work with an interdisciplinary team. Drs. YA Lussier, V Gardeux, and I Achour supplied the informatic methodology and biomedical phenotypes. Myself and Dr. WW Piegorsch developed the MD metric and bootstrapping procedure to test for significant differential pathway expression. I performed the analyses, designed and executed the simulations, and wrote the first draft of the article. Of particular note, I conceived the experiments whose results are displayed in Table 3. Briefly, the sample pairings were changed to compare single samples across patients (as opposed to case-vs.-baseline within patient). This experiment was a precursor for the third work contained in this dissertation (see Chapter \ref{Chap:ctcs} and Appendix \ref{App:E}).
