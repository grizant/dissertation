\chapter{Developing N-of-1\emph{pathways} Mahalanobis distance} \label{Chap:md}

In the foundational paper, the statistical component employed a Wilcoxon signed-rank test, yielding a P-value and its associated signed z-score (as a proxy for effect size). As an alternative, we introduced the Mahalanobis distance (MD; \cite{Schissler2015}) within pathways to quantify differential pathway expression. Informally speaking, this distance is a covariance-adjusted average, signed (log) fold change of expression from baseline to case conditions. The covariance-variance adjustment accounts for structure observed between the paired expression values within the gene set - mRNA fold changes in a tightly correlated pathway are more heavily weighted. The average MD pathway score was seen to be interpretable on the scale of measurement (fold change of expression) than the Wilcoxon test statistic or transformation of its P-value (z-score). As such, the MD statistical component of N-of-1-\emph{pathways} provided a meaningful improvement in quantifying effect. 

- refer to Appendix A \\
- critical picture\\
- my contribution to the project, mention last table as precursor to CTCs\\ 
- contribution to the field\\
