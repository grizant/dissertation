\chapter{CTCs} \label{Chap:ctcs}


\section{Summary of ...} \label{sec:ctcssummary}

- refer to Appendix B
- critical picture\\
- my contribution to the project\\
- contribution to the field\\

%Focus here is on the quantal data setting, with a binomial likelihood. Previous parametric presentations for modeling $R(d)$ have focused on a suite of eight different functions \citep{nauf09,whba09b,pian13}, corresponding to popular choices in the U.S. EPA's BMDS software \citep{davi12}. These eight models, as well as any constraints or bounds on the parameters, are reproduced in Table \ref{dose_response}.
%
%\begin{table}[h!]
%\begin{center}
%\caption{Risk functions for common quantal-response models}\vspace{8pt}
%\label{dose_response}
%\scalebox{0.85}{
%\begin{tabular}{llll}
%\hline
%\textbf{Code} &\textbf{Name} &\textbf{Risk Function} {\boldmath $R(d)$}& \textbf{Constraints} \\
%\hline
%M1&Logistic&$\frac{1}{1+\exp(-\beta_0-\beta_1d)}$&\\
%\\
%M2&Probit&$\Phi(\beta_0+\beta_1d)$&\\
%\\
%M3&Quantal-Linear&$1-\exp(-\beta_0-\beta_1d)$&$\beta_0\ge0$, $\beta_1\ge0$\\
%\\
%M4&Quantal-Quadratic&$\gamma_0+(1-\gamma_0)(1-\exp(-\beta_1d^2))$&$0\le\gamma_0<1$, $\beta_1\ge0$\\
%\\
%M5&Two-Stage&$1-\exp(-\beta_0-\beta_1d-\beta_2d^2)$&$\beta_0,\beta_1,\beta_2\ge0$\\
%\\
%M6&Log-Logistic&$\gamma_0+\frac{1-\gamma_0}{1+\exp(-\beta_0-\beta_1\ln(d))}$&$0\le\gamma_0<1$, $\beta_1\ge0$\\
%\\
%M7&Log-Probit&$\gamma_0+(1-\gamma_0)\Phi(\beta_0+\beta_1\ln(d))$&$0\le\gamma_0<1$, $\beta_1\ge0$\\
%\\
%M8&Weibull&$\gamma_0+(1-\gamma_0)(1-\exp(-e^{\beta_0}d^{\beta_1}))$&$0\le\gamma_0<1$, $\beta_1\ge1$\\
%\hline
%\end{tabular}
%}
%\end{center}
%\end{table}
%
%\noindent Notice in the table that models M$_1$--M$_4$ employ only two unknown parameters, while models M$_5$--M$_8$ employ three.
%
%From the risk functions in Table \ref{dose_response}, the extra risk functions (for definition, see Chapter \ref{Chap:Intro}) for each model are derived. As mentioned earlier, the BMD is the smallest positive solution when solving for $d$ in the equation constructed by setting the extra risk function to equal the BMR. Recall that BMD is denoted as $\xi$. The extra risk functions and the corresponding $\xi$'s for all eight models in Table \ref{dose_response} are shown in Table \ref{extraBMD}:
%
%\begin{table}[h!]
%\begin{center}
%\caption{Extra Risk functions and BMDs for quantal-response models. Note: BMR$\in (0,1)$ is the benchmark response and BMD is the benchmark dose.}\vspace{8pt}
%\label{extraBMD}
%\scalebox{0.85}{
%\begin{tabular}{llll}
%\hline
%\textbf{Code} &\textbf{Name}&\textbf{Extra risk Function}, {\boldmath $R_E(d)$}& \textbf{BMD}, ${\boldsymbol \xi}$ \\
%\hline
%M1&Logistic&$\frac{1-\exp(-\beta_1d)}{1+\exp(-\beta_0-\beta_1d)}$&$\frac{\ln\left(\frac{1+BMR e^{-\beta_0}}{1-BMR}\right)}{\beta_1}$\\
%\\
%M2&Probit&$\frac{\Phi(\beta_0+\beta_1d)-\Phi(\beta_0)}{1-\Phi(\beta_0)}$&$\frac{\Phi^{-1}\{[1-\Phi(\beta_0)]BMR+\Phi(\beta_0)\}-\beta_0}{\beta_1}$\\
%M3&Quantal-Linear&$1-\exp(-\beta_1d)$&$\frac{-\ln(1-BMR)}{\beta_1}$\\
%\\
%M4&Quantal-Quadratic&$1-\exp(-\beta_1d^2)$&$\left(-\frac{\ln(1-BMR)}{\beta_1}\right)^{\frac{1}{2}}$\\
%\\
%M5&Two-Stage&$1-\exp(-\beta_1d-\beta_2d^2)$&$\frac{-\beta_1+\sqrt{\beta_1^2-4\beta_2\ln(1-BMR)}}{2\beta_2}$\\
%\\
%M6&Log-Logistic&$\frac{1}{1+\exp(-\beta_0-\beta_1\ln(d))}$&$\exp\left(\frac{\ln\left(\frac{BMR}{1-BMR}\right)-\beta_0}{\beta_1}\right)$\\
%\\
%M7&Log-Probit&$\Phi(\beta_0+\beta_1\ln(d))$&$\exp\left(\frac{\Phi^{-1}(BMR)-\beta_0}{\beta_1}\right)$\\
%\\
%M8&Weibull&$1-\exp(-e^{\beta_0}d^{\beta_1})$&$\exp\left(\frac{\ln(-\ln(1-BMR))-\beta_0}{\beta_1}\right)$\\
%\hline
%\end{tabular}
%}
%\end{center}
%\end{table}
%

%\section{Reparameterization for quantal dose-response models}\label{sec:Reparameterization}
%
%\indent\indent  As mentioned in \S \ref{sec:Intro litreview}, in applications of Bayesian benchmark analysis to quantal data, objective and sometimes improper forms for the prior p.d.f. $\pi(\boldsymbol{\theta})$ are common, as indicated earlier. These typically appear as diffuse Gaussian priors on the $\beta$-parameters. From this, the joint posterior distribution for $\boldsymbol{\theta}$ is obtained using Bayes formula. An advantage here is that objective priors are usually easy to apply: although they generally lead to intractable integrals, computer intensive operations such as Markov chain Monte Carlo (McMC) methods can produce a sample from the joint posterior of $\boldsymbol{\theta}$ \citep{roca11}. If the sample is sufficiently large and stable, the output can be used to approximate the posterior, from which inferences on the BMD may be conducted. As we suggest above, however, a disadvantage is that $\beta$-parameters may have unclear subject-matter interpretations if those parameters are not target quantities of interest. If informative prior information were available on the risk-analytic quantities under study, the ambiguous interpretation(s) of these traditional, regression-type parameterizations makes incorporation of such information more difficult. This may hinder effective application of the Bayesian approach in this benchmark setting.
%
%For benchmark risk analysis, at least, it is plausible that substantive prior knowledge is available, but not in the form of information on a regression-type $\beta$-parameter. Instead, a risk assessor, toxicologist, or other domain expert would typically have prior knowledge about the target parameter, the BMD, and possibly also about other application-specific values such as the risk at certain doses. In order to conveniently derive and quantify this knowledge, parameterizations that are most familiar to the expert should be favored \citep{grie88}.
%
%The goal is to utilize the potential of the domain expert's prior knowledge for making inferences on the BMD.  To do so, we reparameterize the risk function $R(d)$ in terms of meaningful parameters whose prior distributions are more intuitive to elicit in practice.  The reparameterization strategy is not new, even in benchmark analysis; e.g., \citet[\S14.3.4]{papo05} re-expressed the quantal-linear model in terms of the BMD to facilitate construction of BMDLs (under a frequentist framework). Following on their lead, we reformulate $\boldsymbol{\theta}$ in terms of well-understood risk-analytic quantities: for the dose-response models with two parameters in Table \ref{dose_response}, they are reparameterized in terms of the target value, BMD (denoted in the sequel as $\xi$) and the background risk, say, $\gamma_0 = R(0)$. Thus $\boldsymbol{\theta}$ becomes the vector $[\xi~ \gamma_0]{}^\text{\scriptsize T}$.
%
%For the models with three unknown parameters in Table \ref{dose_response}, they are reparameterized with $\xi$, $\gamma_0 = R(0)$, and a parameter $\gamma_1$ defined as $R(d_\ell)$ for some non-zero dose level $d_\ell$. Unless otherwise specified, $d_\ell$ is set to the highest dose, so $\gamma_1 = R(d_m)$. Thus, $\boldsymbol{\theta} = [\xi~\gamma_0\;\gamma_1]{}^\text{\scriptsize T}$.  The latter two quantities are technically nuisance parameters as far as the BMD is concerned, but one or both are nonetheless likely to be associated with non-trivial prior information; e.g., historical control data may inform $\gamma_0 = R(0)$ \citep{whba12,shao12}. The mathematical developments for the quantal-response models in Table \ref{dose_response} are as follows.
%
%\section{Reparameterizing the Logistic Model (M1)}\label{reM1}
%
%The risk function for the logistic model from Table \ref{dose_response} is
%
%\begin{equation}\label{logistic}
%R(d)=\frac{1}{1+e^{-\beta_0-\beta_1d}}.
%\end{equation}
%%
%Clearly $\gamma_0=R(0)=\frac{1}{1+e^{-\beta_0}}$. Our goal is to rewrite $\beta_0$ and $\beta_1$ in terms of
%$\gamma_0$ and $\xi$=BMD. Begin with $\gamma_0$. After some simple algebra we find
%
%\begin{equation}\label{beta01}
%\beta_0=\log\left(\frac{\gamma_0}{1-\gamma_0}\right).
%\end{equation}
%%
%Next, \citet{wick11} shows that the BMD for the logistic model at a fixed BMR $\in (0,1)$ is
%%
%\begin{equation}\label{bmd1}
%\xi=\frac{\ln\left(\frac{1+\mbox{\scriptsize BMR} e^{-\beta_0}}{1-\mbox{\scriptsize BMR}}\right)}{\beta_1}.
%\end{equation}
%If we solve Equation (\ref{bmd1}) for $\beta_1$, we find
%%
%\begin{equation}\label{gamma01}
%\beta_1=\frac{1}{\xi}\log\left(\frac{1+e^{-\beta_0}\mbox{BMR}}{1-\mbox{BMR}}\right).
%\end{equation}
%%
%Substitute (\ref{beta01}) into (\ref{gamma01}) so that
%%
%\begin{equation}\label{beta11}
%\beta_1=\frac{1}{\xi}\log\left(\frac{1+e^{-\mbox{\scriptsize logit}(\gamma_0)}\cdot \mbox{BMR}}{1-\mbox{BMR}}\right).
%\end{equation}
%%
%This defines $\beta_0$ and $\beta_1$ in terms of $\gamma_0$ and $\xi$. The corresponding, reparameterized risk function is obtained by substituting these expressions into Equation (\ref{logistic}):
%\begin{eqnarray}\label{risk1}
%\nonumber R(d)&=&\frac{1}{1+\exp\left[-\mbox{logit}(\gamma_0)-\frac{1}{\xi}\log\left(\frac{1+e^{-\mbox{\tiny logit}(\gamma_0)}\cdot \mbox{\scriptsize BMR}}{1-\mbox{\scriptsize BMR}}\right)\cdot d\right]} \\
%&=&\left\{1+\exp\left[-\mbox{logit}(\gamma_0)-\frac{1}{\xi}\log\left(\frac{1+e^{-\mbox{\scriptsize logit}(\gamma_0)}\cdot \mbox{\small BMR}}{1-\mbox{\small BMR}}\right)\cdot d\right]\right\}^{-1}.
%\end{eqnarray}
%
%\section{Reparameterizing the Probit Model (M2)}\label{reM2}
%
%The risk function for the probit model from Table \ref{dose_response} is
%
%\begin{equation}\label{probit}
%R(d)=\Phi(\beta_0+\beta_1d).
%\end{equation}
%%
%Clearly $\gamma_0=R(0)=\Phi(\beta_0)$. Our goal is to rewrite $\beta_0$ and $\beta_1$ in terms of
%$\gamma_0$ and $\xi$=BMD. Begin with $\gamma_0$. We have
%
%\begin{equation}\label{beta02}
%\beta_0=\Phi^{-1}(\gamma_0).
%\end{equation}
%%
%Next, \citet{wick11} shows that the BMD for the probit model at a fixed BMR $\in (0,1)$ is
%%
%\begin{equation}\label{bmd2}
%\xi=\frac{\Phi^{-1}\{\mbox{BMR}[1-\Phi(\beta_0)]+\Phi(\beta_0)\}-\beta_0}{\beta_1}.
%\end{equation}
%If we solve Equation (\ref{bmd2}) for $\beta_1$, we find
%%
%\begin{equation}\label{gamma02}
%\beta_1=\frac{\Phi^{-1}\{\mbox{BMR}[1-\Phi(\beta_0)]+\Phi(\beta_0)\}-\beta_0}{\xi}.
%\end{equation}
%%
%Substitute (\ref{beta02}) into (\ref{gamma02}) so that
%%
%\begin{equation}\label{beta12}
%\beta_1=\frac{\Phi^{-1}\{\mbox{BMR}[1-\gamma_0]+\gamma_0\}-\Phi^{-1}(\gamma_0)}{\xi}.
%\end{equation}
%%
%This defines $\beta_0$ and $\beta_1$ in terms of $\gamma_0$ and $\xi$. The corresponding, reparameterized risk function is obtained by substituting these expressions into Equation (\ref{probit}):
%\begin{eqnarray}\label{risk2}
%\nonumber R(d)&=&\Phi(\beta_0+\beta_1d)\\
%&=&\Phi\left\{\Phi^{-1}(\gamma_0)+\frac{\Phi^{-1}\{\mbox{BMR}[1-\gamma_0]+\gamma_0\}-\Phi^{-1}(\gamma_0)}{\xi}d\right\}.
%\end{eqnarray}
%
%\section{Reparameterizing the Quantal Linear Model (M3)}\label{reM3}
%
%The risk function for the quantal linear model from Table \ref{dose_response} is
%
%\begin{equation}\label{quantallinear}
%R(d)=1-\exp(-\beta_0-\beta_1d).
%\end{equation}
%%
%Clearly $\gamma_0=R(0)=1-\exp(-\beta_0)$. Our goal is to rewrite $\beta_0$ and $\beta_1$ in terms of
%$\gamma_0$ and $\xi$=BMD. Begin with $\gamma_0$. We have
%
%\begin{equation}\label{beta03}
%\beta_0=-\log(1-\gamma_0).
%\end{equation}
%%
%Next, \citet{wick11} shows that the BMD for the quantal-linear model at a fixed BMR $\in (0,1)$ is
%%
%\begin{equation}\label{bmd3}
%\xi=\frac{-\log(1-\mbox{BMR})}{\beta_1}.
%\end{equation}
%If we solve Equation (\ref{bmd3}) for $\beta_1$, we find
%%
%\begin{equation}\label{beta13}
%\beta_1=\frac{-\log(1-\mbox{BMR})}{\xi}.
%\end{equation}
%%
%This defines $\beta_0$ and $\beta_1$ in terms of $\gamma_0$ and $\xi$. The corresponding, reparameterized risk function is obtained by substituting these expressions into Equation (\ref{quantallinear}):
%\begin{eqnarray}\label{risk3}
%R(d)=1-\exp\left(\log(1-\gamma_0)+\frac{\log(1-\mbox{BMR})}{\xi}d\right).
%\end{eqnarray}
%
%\section{Reparameterizing the Quantal Quadratic Model (M4)}\label{reM4}
%
%The risk function for the quantal quadratic model from Table \ref{dose_response} is
%
%\begin{equation}\label{quantalquadratic}
%R(d)=\gamma_0+(1-\gamma_0)[1-\exp(-\beta_1d^2)].
%\end{equation}
%%
%Clearly $\gamma_0=R(0)$. Our goal is to rewrite $\beta_1$ in terms of
%$\gamma_0$ and $\xi$=BMD.
%
%Next, \citet{wick11} shows that the BMD for the quantal-quadratic model at a fixed \mbox{BMR} $\in (0,1)$ is
%%
%\begin{equation}\label{bmd4}
%\xi=\left(-\frac{\log(1-\mbox{BMR})}{\beta_1}\right)^{\frac{1}{2}}.
%\end{equation}
%If we solve Equation (\ref{bmd4}) for $\beta_1$, we find
%%
%\begin{equation}\label{beta14}
%\beta_1=-\frac{\log(1-\mbox{BMR})}{\xi^2}.
%\end{equation}
%%
%Therefore, the corresponding, reparameterized risk function is obtained by substituting (\ref{beta14}) into Equation (\ref{quantalquadratic}):
%\begin{eqnarray}\label{risk4}
%\nonumber R(d)&=&\gamma_0+(1-\gamma_0)[1-\exp(-\beta_1d^2)]\\
%&=&\gamma_0+(1-\gamma_0)\left(1-\exp\left\{\frac{\log(1-\mbox{BMR})}{\xi^2}d^2\right\}\right).
%\end{eqnarray}
%
%\section{Reparameterizing the Multi-stage (Two-stage) Model (M5)}\label{reM5}
%
%The risk function for the two-stage model from Table \ref{dose_response} is
%
%\begin{equation}\label{oldmulti}
%R(d)=1-\exp(-\beta_0-\beta_1d-\beta_2d^2).
%\end{equation}
%%
%With this, we define $\gamma_0 = R(0) = 1-\exp(-\beta_0)$, hence, $\exp(-\beta_0)=1-\gamma_0$. Substitute $\exp(-\beta_0)=1-\gamma_0$ into the risk function to find
%%
%\begin{eqnarray}\label{newmulti}
%\nonumber R(d)&=&1-\exp(-\beta_0)\exp(-\beta_1d-\beta_2d^2)\\
%\nonumber &=&1-(1-\gamma_0)\exp(-\beta_1d-\beta_2d^2)\\
%\nonumber &=&1+(1-\gamma_0)(1-1-\exp(-\beta_1d-\beta_2d^2))\\
%\nonumber &=&1+(1-\gamma_0)(1-\exp(-\beta_1d-\beta_2d^2))-(1-\gamma_0)\\
%&=&\gamma_0+(1-\gamma_0)(1-\exp(-\beta_1d-\beta_2d^2)).
%\end{eqnarray}
%%
%Next, we reexpress $\beta_1$ and $\beta_2$ in terms of $\xi$ , $\gamma_0$ and $\gamma_1$. Begin with $\gamma_1$, expressed in terms of $\gamma_0$, $\beta_1$ and $\beta_2$. From the definition of an $\gamma_1$, we know
%%
%\[\gamma_1=\gamma_0+(1-\gamma_0)(1-\exp(-\beta_1d_m-\beta_2d_m^2)),\]
%where $d_m$ denotes the $m^{\scriptsize \mbox{th}}$ (the highest) dose level.
%
%Denote $\Gamma_5=\log\left(\frac{1-\gamma_1}{1-\gamma_0}\right)$. Recall that $0<\gamma_1<1$. After some algebra, we obtain
%
%\begin{equation}\label{GM5}
%\beta_2d_m^2+\beta_1d_m+\Gamma_5=0.
%\end{equation}
%
%Next, \cite[Ex. 4.12]{piba05} show that the benchmark dose for the two-stage model can be found to be
%%
%\begin{equation}\label{bmd5}
%\xi=\frac{-\beta_1+\sqrt{\beta_1^2-4\beta_2\ln(1-\mbox{BMR})}}{2\beta_2}.
%\end{equation}
%
%\noindent Now, denote  $C_5=-\ln(1-\mbox{BMR})$, from Equation (\ref{bmd5}), we have
%\begin{equation}\label{xi5}
%\beta_2\xi^2+\beta_1\xi-C_5=0 .
%\end{equation}
%%
%If we solve Equation (\ref{GM5}) for $\beta_1$, then we have $\beta_1=\frac{-\Gamma_5-\beta_2d_m^2}{d_m}$. Substitute the result into (\ref{xi5}) to find
%%
%\begin{equation}\label{beta25}
%\beta_2=\frac{\Gamma_5\xi+C_5 d_m}{\xi d_m(\xi-d_m)}.
%\end{equation}
%%
%Similarly, if we solve Equation (\ref{GM5}) for $\beta_2$, then we have $\beta_2=\frac{-\Gamma_5-\beta_1d_m}{d_m^2}$. Substitute the result into (\ref{xi5}) to find
%%
%\begin{equation}\label{beta15}
%\beta_1=\frac{C_5d_m^2+\Gamma_5\xi^2}{\xi d_m(d_m-\xi)}.
%\end{equation}
%
%So far we have rewritten $\beta_1$ and $\beta_2$ in terms of $\Gamma_5$ (or $\gamma_0$ and $\gamma_1$) and $\xi$ . Therefore, the reparameterized risk function is obtained by substituting these expressions into Equation (\ref{newmulti}).  This produces
%%
%\begin{eqnarray}\label{risk5}
%\nonumber R(d)&=&\gamma_0+(1-\gamma_0)\left(1-\exp\left(-\frac{C_5d_m^2+\Gamma_5\xi^2}{\xi d_m(d_m-\xi)}d-\frac{\Gamma_5\xi+C_5 d_m}{\xi d_m(\xi-d_m)}d^2\right)\right)\\
%\nonumber &=&\gamma_0+(1-\gamma_0)\left(1-\exp\left(\frac{C_5d_m^2 d+\Gamma_5 \xi^2 d}{\xi d_m(\xi-d_m)}-\frac{\Gamma_5\xi d^2+C_5 d_m d^2}{\xi d_m(\xi-d_m)}\right)\right)\\
%\nonumber &=&\gamma_0+(1-\gamma_0)\left(1-\exp\left(\frac{C_5d_m d (d_m-d)+\Gamma_5 \xi d (\xi-d)}{\xi d_m(\xi-d_m)}\right)\right),\\
%\end{eqnarray}
%where $\Gamma_5=\log\left(\frac{1-\gamma_1}{1-\gamma_0}\right)$ and $C_5=-\log(1-\mbox{BMR})$.
%
%\section{Reparameterizing the Log-logistic Model (M6)}\label{reM6}
%
%The risk function for the log-logistic model is
%%
%\begin{equation}\label{oldloglogistic}
%R(d)=\gamma_0+(1-\gamma_0)\left[1+\exp(-\beta_0 - \beta_1\ln d)\right]^{-1}.
%\end{equation}
%%
%Clearly, $\lim_{d\rightarrow0}R(d)=\gamma_0$.  As above, our goal is to rewrite $\beta_0$ and $\beta_1$ in terms of $\gamma_0$, $\xi$ and $\gamma_1$.
%Begin with $\gamma_1$, which is defined by the following equation:
%%
%\[\gamma_1=\gamma_0+(1-\gamma_0)\left[1+\exp(-\beta_0 - \beta_1\ln d_m)\right]^{-1},\]
%where $d_m$ denotes the $m^{\scriptsize \mbox{th}}$ (the highest) dose level.
%
%Denote $\Gamma_6=\frac{1-\gamma_1}{1-\gamma_0}$, after some simple algebra we have
%
%\begin{equation}\label{GM6}
%\beta_0+\beta_1\log d_m+\Gamma_6=0.
%\end{equation}
%
%Next, \citet{wick11} shows that the BMD for the log-logistic model at a fixed BMR $\in (0,1)$ is
%%
%\begin{equation}\label{bmd6}
%\xi=\exp\left(\frac{\mbox{logit}(\mbox{BMR})-\beta_0}{\beta_1}\right).
%\end{equation}
%%
%Denote $C_6=\log\left(\frac{\small \mbox{BMR}}{\small 1-\mbox{BMR}}\right)$, after some simple algebra we have
%\begin{equation}\label{xi6}
%\beta_0+\beta_1\log\xi-C_6=0.
%\end{equation}
%If we solve Equation \eqref{GM6} for $\beta_0$, we find $\beta_0=-\beta_1\log d_m-\Gamma_6$, substitute the result into Equation \eqref{xi6} to find
%%
%\begin{equation}\label{beta16}
%\beta_1=\frac{C_6+\Gamma_6}{\log\xi-\log d_m}.
%\end{equation}
%%
%Similarly, if we solve Equation \eqref{GM6} for $\beta_1$, we find $\beta_1=\frac{-\beta_0-\Gamma_6}{\log d_m}$, substitute the result into Equation \eqref{xi6} to find
%%
%\begin{equation}\label{xi6}
%\beta_0=\frac{C_6\log d_m+\Gamma_6 \log\xi}{\log d_m-\log\xi}.
%\end{equation}
%%
%This defines $\beta_0$ and $\beta_1$ in terms of $\Gamma_6$ (or $\gamma_0$ and $\gamma_1$) and $\xi$. The corresponding, reparameterized risk function is obtained by substituting these expressions into Equation (\ref{oldloglogistic}):
%\begin{eqnarray}\label{risk6}
%\nonumber R(d)&=&\gamma_0+(1-\gamma_0)\left[1+\exp\left(-\frac{C_6\log d_m+\Gamma_6 \log\xi}{\log d_m-\log\xi} -\frac{C_6+\Gamma_6}{\log\xi-\log d_m}\log d\right)\right]^{-1}\\
%\nonumber &=&\gamma_0+(1-\gamma_0)\left[1+\exp\left(\frac{C_6(\log d_m-\log d)+\Gamma_6(\log\xi-\log d)}{\log\xi-\log d_m}\right)\right]^{-1}\\
%\end{eqnarray}
%where $\Gamma_6=\log\left(\frac{1-\gamma_1}{1-\gamma_0}\right)$ and $C_6=\log\left(\frac{\small \mbox{BMR}}{\small 1-\mbox{BMR}}\right)$.
%
%\section{Reparameterizing the Log-probit Model (M7)}\label{reM7}
%
%The risk function for the log-probit model is
%%
%\begin{equation}\label{oldlogprobit}
%R(d)=\gamma_0+(1-\gamma_0)\Phi(\beta_0+\beta_1\ln d).
%\end{equation}
%%
%Clearly, $\lim_{d\rightarrow0}R(d)=\gamma_0$.  As above, our goal is to rewrite $\beta_0$ and $\beta_1$ in terms of $\gamma_0$, $\xi$ and $\gamma_1$. Begin with $\gamma_1$, which is defined by the following equation:
%%
%\[\gamma_1=\gamma_0+(1-\gamma_0)\Phi(\beta_0+\beta_1\ln d_m),\]
%where $d_m$ denotes the $m^{\scriptsize \mbox{th}}$ (the highest) dose level.
%
%Denote $\Gamma_7=\Phi^{-1}\left(\frac{\gamma_1-\gamma_0}{1-\gamma_0}\right)$, after some simple algebra we have
%
%\begin{equation}\label{GM7}
%\beta_0+\beta_1\log d_m-\Gamma_7=0.
%\end{equation}
%
%Next, derivations similar to those in model M6 show that the BMD for the log-probit model at a fixed BMR $\in (0,1)$ takes the same general form:
%%
%\begin{equation}\label{bmd7}
%\xi = \exp\left(\frac{C_7-\beta_0}{\beta_1}\right),
%\end{equation}
%%
%Denote $C_7=\Phi^{-1}(\mbox{BMR})$, after some simple algebra we have
%
%\begin{equation}\label{xi7}
%\beta_0+\beta_1\log\xi-C_7=0.
%\end{equation}
%%
%
%If we solve Equation \eqref{GM7} for $\beta_0$, we find $\beta_0=\Gamma_7-\beta_1\log d_m$, substitute the result into Equation \eqref{xi7} to find
%\begin{equation}\label{beta17}
%\beta_1=\frac{C_7-\Gamma_7}{\log\xi-\log d_m}.
%\end{equation}
%
%Similarly if we solve Equation (\ref{GM7}) for $\beta_1$, we find $\beta_1=\frac{\Gamma_7-\beta_0}{\log d_m}$, substitute the result into Equation \eqref{xi7} to find
%
%\begin{equation}\label{beta07}
%\beta_0=\frac{C_7\log d_m-\Gamma_7\log\xi}{\log d_m-\ln\xi}.
%\end{equation}
%
%This defines $\beta_0$ and $\beta_1$ in terms of $\Gamma_7$ (or $\gamma_0$ and $\gamma_1$) and $\xi$.  The corresponding, reparameterized risk function is obtained by substituting these expressions into Equation (\ref{oldlogprobit}):
%%
%\begin{eqnarray}\label{risk7}
%\nonumber R(d)&=&\gamma_0+(1-\gamma_0)\Phi\left(\frac{C_7\log d_m-\Gamma_7\log\xi}{\log d_m-\ln\xi}+\frac{C_7-\Gamma_7}{\log\xi-\log d_m}\log d\right)\\
%\nonumber &=&\gamma_0+(1-\gamma_0)\Phi\left(\frac{C_7\log d_m-\Gamma_7\log\xi+\Gamma_7\log d-C_7\log d}{\log d_m-\log\xi}\right)\\
%\nonumber &=&\gamma_0+(1-\gamma_0)\Phi\left(\frac{C_7(\log d_m-\log d)+\Gamma_7(\log d-\log\xi)}{\log d_m-\log\xi}\right),\\
%\end{eqnarray}
%where $\Gamma_7=\Phi^{-1}\left(\frac{\gamma_1-\gamma_0}{1-\gamma_0}\right)$ and $C_7=\Phi^{-1}(\mbox{BMR})$.
%
%\section{Reparameterizing the Weibull Model (M8)}\label{reM8}
%
%The risk function for the Weibull model is
%%
%\begin{equation}\label{oldweibull}
%R(d)=\gamma_0+(1-\gamma_0)\left[1-\exp(-\exp(\beta_0 + \beta_1\ln d))\right].
%\end{equation}
%%
%Clearly, $\lim_{d\rightarrow0}R(d)=\gamma_0$.  As above, our goal is to rewrite $\beta_0$ and $\beta_1$ in terms of $\gamma_0$, $\xi$ and $\gamma_1$. Begin with $\gamma_1$, which is defined by the following equation:
%%
%\[\gamma_1=\gamma_0+(1-\gamma_0)\left[1-\exp(-\exp(\beta_0 + \beta_1\ln d_m))\right],\]
%where $d_m$ denotes the $m^{\scriptsize \mbox{th}}$ (the highest) dose level.
%
%Denote $\Gamma_8=\log\left(-\log\left(\frac{1-\gamma_1}{1-\gamma_0}\right)\right)$, after some simple algebra we have
%\begin{equation}\label{GM8}
%\beta_0+\beta_1\log d_m-\Gamma_8=0.
%\end{equation}
%
%Next, derivations similar to those in model M6 show that the BMD for the log-probit model at a fixed BMR $\in (0,1)$ takes the same general form:
%%
%\begin{equation}\label{bmd8}
%\xi = \exp\left(\frac{C_8-\beta_0}{\beta_1}\right),
%\end{equation}
%%
%Denote $C_8=\log(-\log(1-\mbox{BMR}))$, after some simple algebra we have
%
%\begin{equation}\label{xi8}
%\beta_0+\beta_1\log\xi-C_8=0.
%\end{equation}
%%
%
%If we solve Equation \eqref{GM8} for $\beta_0$, we find $\beta_0=\Gamma_8-\beta_1\log d_m$, substitute the result into Equation \eqref{xi8} to find
%\begin{equation}\label{beta18}
%\beta_1=\frac{C_8-\Gamma_8}{\log\xi-\log d_m}.
%\end{equation}
%
%Similarly if we solve Equation (\ref{GM8}) for $\beta_1$, we find $\beta_1=\frac{\Gamma_8-\beta_0}{\log d_m}$, substitute the result into Equation \eqref{xi8} to find
%
%\begin{equation}\label{beta08}
%\beta_0=\frac{C_8\log d_m-\Gamma_8\log\xi}{\log d_m-\ln\xi}.
%\end{equation}
%
%This defines $\beta_0$ and $\beta_1$ in terms of $\Gamma_8$ (or $\gamma_0$ and $\gamma_1$) and $\xi$.  The corresponding, reparameterized risk function is obtained by substituting these expressions into Equation (\ref{oldweibull}):
%%
%\begin{eqnarray}\label{risk8}
%\nonumber &&R(d)\\
%\nonumber &=&\gamma_0+(1-\gamma_0)\left[1-\exp\left(-\exp\left(\frac{C_8\log d_m-\Gamma_8\log\xi}{\log d_m-\ln\xi}+\frac{C_8-\Gamma_8}{\log\xi-\log d_m}\log d\right)\right)\right]\\
%\nonumber &=&\gamma_0+(1-\gamma_0)\left[1-\exp\left(-\exp\left(\frac{C_8\log d_m-\Gamma_8\log\xi+\Gamma_8\log d-C_8\log d}{\log d_m-\log\xi}\right)\right)\right]\\
%\nonumber &=&\gamma_0+(1-\gamma_0)\left[1-\exp\left(-\exp\left(\frac{C_8(\log d_m-\log d)+\Gamma_8(\log d-\log\xi)}{\log d_m-\log\xi}\right)\right)\right],\\
%\end{eqnarray}
%where $\Gamma_8=\log\left(-\log\left(\frac{1-\gamma_1}{1-\gamma_0}\right)\right)$ and $C_8=\log(-\log(1-\mbox{BMR}))$.
%
%\vspace{1pc}
%
%These various reparameterized forms explicitly display the eight dose-response functions from Table \ref{dose_response} in terms of model parameters pertinent to a toxicologist or risk assessor. These reparameterizations present more burdensome notation for $R(d)$. The explicit incorporation of the target parameter $\xi$ and well-understood quantities such as $\gamma_0$ and $\gamma_1$ allows us, however, to formulate a more application-oriented hierarchical model, from which to produce inferences on $\xi$. 