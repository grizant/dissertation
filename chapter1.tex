\chapter{Introduction}\label{Chap:Intro}

\section{Article-based dissertation and organization}\label{sec:org}
format of this document

\section{Single-subject transcriptome analytics}\label{sec:nof1}

\indent \indent definitions, motivation, and application

% \indent \indent Quantitative risk assessment involves estimation of the severity and likelihood of adverse responses associated with exposure to hazardous stimuli. With this context, one can, e.g., assess many forms of risk such as biological disease risks \citep{wake08}, environmental health risks \citep{ster08}, ecological risks \citep{fox06}, economic loss due to natural disasters \citep{boru06}, industrial or engineering risks such as nuclear power plant failures \citep{grim02}, etc.  This dissertation will exclusively focus on risks to biological (including human) or ecological systems. In these situations, `risk' can be defined as follows.
%
%\begin{definition}\label{riskdef}
%Risk, R(d), is defined as the probability of some pre-defined adverse effect, such as death, weight loss, birth defect, cancer or mutation exhibited in a subject exposed to a particular dose level, d, of a hazardous agent.
%\end{definition}
%
%Definition \ref{riskdef} defines risk in terms of an adverse effect. Thus, it is reasonable to assume that $R(d)$ is a monotone increasing function of dose. Although seemingly straightforward, this definition contains an important, implicit feature: non-zero risk may exist, even change, for very small levels of $d$. This extends earlier concepts of risk where, at least for many non-cancer endpoints, one assumed that some dose threshold exists below which $R(d) = 0$. By modeling the risk more formally, however, a richer variety of possible dose-response functions and consequent statistical machinery becomes available \citep{krry1981}. Notice that the unknown quantity $R(0)$ represents the risk to which all subjects in a population are exposed. To correct for this spontaneous risk of response, additional risk and extra risk functions frequently are used in risk estimation \citep[ \S 4.2.1]{piba05}. Defined formally, these are as follows:
%
%\begin{definition}\label{addriskdef}
%Additional risk, $R_A(d)$, is the risk beyond that of the control
%(background, spontaneous) level; that is:
%\begin{equation}
%R_A(d)=R(d)-R(0).
%\end{equation}
%\end{definition}
%
%\begin{definition}\label{exriskdef}
%Extra risk, $R_E(d)$, is the additional risk among subjects who on average
%would not have responded under control conditions:
%\begin{equation}
%R_E(d)=\frac{R_A(d)}{1-R(0)}=\frac{R(d)-R(0)}{1 - R(0)}, \mbox{ where }R(0) < 1.
%\end{equation}
%\end{definition}
%%
%Notice that when the background risk is 0, i.e., $R(0)=0$, the additional risk and extra risk are both equal to $R(d)$. Moreover, if $0\le R(0)<1$ ($R(0)$ seldom equals 1 but it may equal 0), then $R_E(d)\ge R_A(d)$; in other words, extra risk can be thought of as the additional risk after inflation to account for control-condition non-response. Together, $R_E(d)$ and $R_A(d)$ are known as forms of \emph{excess risk}.
%
%Quantitative risk assessment in public health or environmental applications is usually broken down into four fundamental steps \citep{ster02}. First is \emph{hazard identification}, where an agent, substance, or other environmental stimulus induces detrimental outcomes in some industrial, occupational, public health, or ecological setting. The detrimental outcomes envisioned in this dissertation may be cancer, mutation, birth defect, ecosystem damage, etc. Second is \emph{stimulus/dose-response assessment}, where any quantifiable relationship between the agent and the detrimental outcome is modeled and estimated. Here the term \emph{dose} is used as a generic label for any quantification of a hazardous exposure. Third is \emph{exposure assessment}, where the extent, frequency, and duration of the exposure before and after application of regulatory measures are determined. Last is \emph{risk characterization} in which the risk analyst incorporates information from the previous three stages into a single assessment of the overall risk due to exposure to the hazardous agent. Using these four steps as a basic paradigm for the risk assessment process, it is assumed herein that identification of the hazard has been addressed and that the focus will be on dose-response assessment and quantitative risk estimation.
%

\section{N-of-1-\emph{pathways} framework}\label{sec:nof1pathways}

\indent\indent three principles

\section{N-of-1-\emph{pathways} Wilcoxon}\label{sec:wilcoxon}

- give the formula (this has to speak to statisticians!!)

\section{Gene set testing}\label{sec:genesets}

- self-contained vs. competitive

%In many cases, epidemiological or human data on hazardous substances are not available or are inadequate for quantitative risk assessment. This may be due to the lack of accurate information on exposure levels or the confounding of risk factors where it is impossible to separate the effects of one hazard from another \citep{krew91, bapo1994}. Thus, to assess the adverse effects of a hazardous agent, bioassays are often conducted on laboratory rodents (mice, rats, etc.) or other biological systems such as aquatic animals or cells in a laboratory culture. Due to the short life span of laboratory animals and to guarantee that a toxic effect will be observable within a reasonably short period of time (say, several weeks or months), the dose levels of the agent(s) are administered at relatively high values. This is true primarily for laboratory animal experiments conducted as screens for certain toxic effects \citep{hase84, hase85, Pieg94}. Unfortunately, the dose-response pattern exhibited at high doses in such a study may not apply in the low-dose region. Many candidate dose-response models may fit data equally well at high doses, but may also yield dramatically different estimates at lower does levels \citep[ \S4.2.2]{piba05}.  The so-called \emph{low-dose extrapolation} problem for estimating risk from high dose levels to lower doses of regulatory interest \citep{brko83} is one of the greatest challenges in quantitative assessment of possible human or ecological risks.
%
%A contemporary approach to low-dose estimation, known as \emph{benchmark risk analysis}, uses the functional specification for $R(d)$ to provide low-dose estimates for risk and/or excess risk. In this approach, the \emph{benchmark dose} at which a predetermined level of risk is attained is estimated by inverting the dose-response relationship. Formal definitions used in benchmark analysis are as follows:
%
%\begin{definition}\label{bmddef}
%The Benchmark Dose (BMD) is an exposure due to a dose of a substance associated with a specified low incidence of risk, generally in the range of $1\%$ to $10\%$, of a health effect; or the dose associated with a specified measure or change of a biological effect.
%\end{definition}
%
%\begin{definition}\label{bmrdef}
%The Benchmark Response (BMR) is a response, generally expressed as excess of background (e.g., extra risk), at which a benchmark dose is desired.
%\end{definition}
%
%For a typical data setting (for example, the quantal data which will be introduced in \S \ref{sec:Intro quantalbinom}), the BMD is usually found by solving for the smallest $d\ge0$ that satisfies $R_E(d)$=BMR.
%
%\begin{definition}\label{bmdldef}
%The Benchmark Dose Lower Limit (BMDL) is a lower one-sided 1-$\alpha$ confidence (or credible) limit on the BMD.
%\end{definition}
%
%It is common to use the BMDL as a point of departure which is then reduced by a set of uncertainty/safety factors to arrive at an acceptable level of human exposure or to otherwise establish human low-exposure guidelines \citep{gayl98}. BMDLs corresponding to BMR = 0.01, 0.05 or 0.10 are most often seen in practice for a given adverse effect \citep{BMD12}. Where needed for clarity, a subscript is added for the BMR level at which each quantity is calculated:  BMD$_{\text{\tiny 100BMR}}$ and BMDL$_{\text{\tiny 100BMR}}$.
%In this fashion, use of BMDs and BMDLs for quantifying and managing risk is growing in both the United States and the European Union \citep{gao01,eu03,oecd06,oecd08}.
%
%\section{Quantal Response Data and Binomial Framework}\label{sec:Intro quantalbinom}
%\indent\indent In risk-analytic dose-response studies, `quantal' data are often observed. With this type of data, the observations are in the form of proportions and the experimental subjects are classified in a binary fashion as either exhibiting or not exhibiting the adverse effect. Such data are very common in carcinogenesis, teratogenesis and mutagenesis studies \citep{crum76,chko89, gayl89, gayl98}.
%
%With quantal data, benchmark analysis is usually performed under a binomial framework. Denote $Y_i$ as the number of responses at the $i^{\mbox{\scriptsize th}}$ dose level, out of $N_i$ subjects tested at that dose $(i = 1, ..., m)$.  The standard statistical model assumes $Y_i\sim \mbox{indep.}\  Binom\bigr(N_i, R(d_i)\bigl)$, where $R(d_i)$ is the risk at dose $d_i$. For generic purposes, denote {\boldmath$\theta$} as an unknown parameter vector that describes $R(d)$.
%Under the independence assumption, the joint probability mass function (p.m.f.)
%for ${\boldsymbol Y}=(Y_1, \ldots, Y_m)^T$ becomes
%%
%\begin{equation}\label{eq:likelhd}
%f(\mbox{\boldmath $Y$}|\mbox{\boldmath $\theta$}) = \prod_{i=1}^{m}f(Y_i|\mbox{\boldmath $\theta$}) = \prod_{i=1}^m{N_i \choose Y_i}R(d_i)^{Y_i}\{1-R(d_i)\}^{N_i-Y_i},
%\end{equation}
%%
%where $f(Y_i|\mbox{\boldmath $\theta$})$ is the individual binomial p.m.f. for each $Y_i$.
%
%Traditionally, estimation of the BMD has been performed via maximum likelihood: viewing the joint p.m.f.~as a likelihood function, $L(\mbox{\boldmath $\theta$}) = f(\mbox{\boldmath $Y$}|\mbox{\boldmath $\theta$})$,  maximizing $L(\mbox{\boldmath $\theta$})$ or $\log\{L(\mbox{\boldmath $\theta$})\}$ produces maximum likelihood estimators (MLEs) $\widehat{\mbox{\boldmath $\theta$}}$ for the parameter vector {\boldmath$\theta$}. The corresponding MLEs $\widehat{R}(d)$ for $R(d)$ and $\widehat{R}_E(d)$ for $R_E(d)$ are then obtained using the invariance property of MLE. Setting $\widehat{R}_E(d) =$ BMR and solving for $d$ yields the MLE, $\widehat{\mbox{BMD}}_{\text{\tiny 100BMR}}$, for the target quantity, BMD. This method is a form of inverse non-linear regression and, except for the use of an excess risk function upon which to base the inversion, is essentially equivalent to estimation of an `effective dose' such as the well-known median effective dose, ED$_{50}$ \citep[\S4.1.1]{piba05}. The corresponding BMDL is then built from the statistical properties of $\widehat{\mbox{BMD}}$; parametric possibilities include Wald-type lower confidence limits \citep{moer04}, profile likelihood limits \citep{crho85}, or appeal to the bootstrap \citep{weni09}.  The latter has also been employed for constructing non-parametric BMDLs \citep{pixi12}.
%
%\section{Hierarchical Bayesian Modeling}\label{sec:Intro HierBayes}
%\indent\indent The maximum likelihood approach mentioned above for calculating BMDs typically involves non-hierarchical models and in particular, frequentist confidence limits for the BMDL.  Different from the frequentist framework, Bayesian methods view this problem in a hierarchical perspective. Here, $\boldsymbol Y$ is modeled conditionally on {\boldmath$\theta$}. If {\boldmath $\theta$} is itself thought to be random, say, $\pi(\boldsymbol\theta)$, application of Bayes rule \citep[ \S1.3]{cabe02} results in a form of \emph{updated} information on {\boldmath$\theta$}:
%%
%\begin{equation}\label{bayesformula}
%\pi({\boldsymbol \theta}|{\boldsymbol Y})=\frac{\pi({\boldsymbol \theta})f({\boldsymbol Y}|{\boldsymbol \theta})}{\int_{\boldsymbol \Theta}\pi({\boldsymbol \theta})f({\boldsymbol Y}|{\boldsymbol \theta})d{\boldsymbol \theta}},
%\end{equation}
%%
%where $f(\boldsymbol Y|\boldsymbol\theta)$ (known as the \emph{likelihood function}) represents the parent distribution indexed by ${\boldsymbol \theta}$, $\pi({\boldsymbol \theta})$ is the \emph{prior density function} for $\boldsymbol \theta$ (the distribution corresponding to it is known as the \emph{prior distribution}), and $\pi({\boldsymbol\theta}|{\boldsymbol Y})$ is called the \emph{posterior density function} for $\boldsymbol\theta$ after updating with the data $\boldsymbol Y$ (the distribution corresponding to this density is known as the \emph{posterior distribution}).
%
%This hierarchical structure for the data and {\boldmath$\theta$} is often viewed from what is known as the \emph{Bayesian formulation} for statistical inference \citep[ \S\S7.2.3, 9.2.4]{cabe02}.  Simply put, a fully specified posterior distribution for $\boldsymbol\theta$ tells a Bayesian all there is to know about $\boldsymbol\theta$; estimation and inferences follow directly from this.
%
%\section{Established Bayesian Benchmark Dose Analysis}\label{sec:Intro litreview}
%\indent\indent For estimating a benchmark dose, the BMD is viewed as one of the unknown parameters in $\boldsymbol\theta$, along with any other nuisance parameters required by the posited dose-response model.  For simplicity, denote $\xi$ as BMD.  Hereupon, emphasis will be restricted to fully parametric Bayesian estimation for and inference on $\xi$.
%
%Established Bayesian benchmark dose analysis methods typically use generalized linear models (GLiM) to model the dose-response relationship under a binomial likelihood. In most models, $\xi$ is not included in the default parameter set; instead, $R(d)$ is usually modeled via a linear predictor:
%\begin{equation}\label{GLiM}
%g(R(d))=\beta_0+\beta_1x_1(d)+\beta_2x_2(d)+\ldots,
%\end{equation}
%where $g(\cdot)$ is called the \emph{link function} which maps the probability value to the whole real line and $x_i(d)$ are functions of $d$ which are monotone on $[0, \infty)$ \citep[ \S3.1.1]{brpr06}. Fully parametric Bayesian benchmark risk analyses with quantal data which have appeared in the literature have employed this form (or variants thereof). The parameters in these models usually don't have pertinent risk analytic interpretations, hence it is difficult to construct pertinent prior distributions for them. As a consequence, the prior distributions for these parameters are often chosen to be `non-informative' or `improper'. (An improper prior has an infinite integral over the range $(-\infty, \infty)$; a non-informative prior is any prior with a very flat density curve, loosely speaking, but the integral over the range $(-\infty, \infty)$ is finite; see \citet[ \S 2.9]{geca04} and \citet[ \S3.2]{gewe05}.) The corresponding hierarchical models lose the ability to intelligently incorporate the analyst's prior knowledge and the inferences based on these priors rely heavily on the input data \citep[ \S2.9]{geca04}. For example, \citet{shsm11} use the well-known logistic model $R(d)=\{1+\exp(-\beta_0-\beta_1d)\}^{-1}$ and the so-called quantal-linear model $R(d)=1-\exp(-\beta_0-\beta_1d)$. The latter is a special case of the more general multi-stage ($k$-stage) model $R(d)=1-\exp(-\beta_0-\beta_1d-\ldots-\beta_kd^k)$. They apply essentially non-informative normal priors for $\beta_0$ and $\beta_1$ for modeling quantal data in a tumorigenicity experiment; further developments appeared in \citet{shsm12}. \citet{shao12} expanded these considerations to the probit model $R(d)=\Phi(\beta_0+\beta_1d)$, where $\Phi(\cdot)$ is the CDF for standard normal distribution, and also introduced a power prior to build historical control information into the hierarchy. (A variety of quantal response models, including the logistic, probit and quantal-linear will be introduced in \S2.1, below.) \citet{wang11} use the logistic model and the probit model with non-informative hierarchical normal, gamma and uniform priors for the $\beta$-parameters to study BMDs for nephrolithiasis in children. \citet{nauf09} use the entire suite of dose-response models in the EPA's Benchmark Dose software, BMDS \citep{davi12}, except for its quantal-quadratic model (see Table \ref{dose_response}, below), with non-informative uniform priors for the $\beta$-parameters to study the effect of a tobacco specific nitrosamine (NNK) on oral cancer. Other fully parametric Bayesian benchmark risk analyses with continuous or count data are also based on forms of generalized linear models. For example, \citet{shgi14} apply non-informative priors for the $\beta$-parameters in four dose-response models for continuous data listed in BMDS to study model uncertainty issue in continuous data benchmark dose analysis. \citet{whba09a} use Poisson regression models with non-informative normal priors for the mean parameters and non-informative inverse gamma priors for the variance parameters to study the effect of NaCl on the reproducibility of \emph{Ceriodaphnia dubia} (a water flea which is used in toxicity testing of waste-water treatment plant effluent water in the United States). \citet{moib06} use similar models with non-informative normal priors for the parameters to study the effect of arsenic in drinking water on lung cancer and bladder cancer. \citet{held04} uses a quadratic regression model with improper priors for the $\beta$-parameters to model the effects of copper toxicity in sea kelp.
%
%Similar to the BMD, an approach to inverse dose estimation is known as the \emph{effective dose} $\rho$. This is defined as the dose that yields a $100\rho\%$ effect over the quantal dose-response curve, for $0<\rho<1$ \citep[ \S4.1.2]{piba05}. Common notation for this is $\mbox{ED}_{100\rho}$. Effective dose analysis is similar to benchmark risk analysis in that both methods invert a function to estimate a dose that quantifies a critical effect. The effective dose is obtained by inverting the risk function, i.e., solving $R(d)=\rho$. By contrast, the BMD as employed here with quantal data solves $R_E(d)=\mbox{BMR}$. For finding a Bayesian $\mbox{ED}_{100\rho}$,  \citet{chen10} uses the logistic model with improper priors for the unknown $\beta$-parameters. \citet{lizh08} use the logistic model with three types of priors (improper, exponential and conjugate) for the unknown $\beta$-parameters. \citet{hu08} use the logistic model with both improper priors and uniform priors for the $\beta$-parameters. \citet{suts97} consider a one-parameter and a two-parameter logistic model with non-informative normal priors and gamma priors for the parameters to derive a Bayesian design in order to minimize the expected posterior variances for the $\mbox{ED}_{100\rho}$. \citet{kuco99} use multi-stage models (see above, with number of stages equaling 1, 2 and 3) with improper priors for the parameters to estimate a Bayesian effective dose. Notably, \citet{hu08} and \citet{suts97} reparameterize their models to include the $\mbox{ED}_{100\rho}$ as a parameter in the model. This provides an opportunity to perform direct inferences through the posterior distribution on the effective dose.
%
%In the above examples, Bayesian estimates of the BMD or ED are usually obtained indirectly. A necessary first step is usually to estimate the traditional model parameters such as $\beta_0$ and $\beta_1$ in (\ref{GLiM}). Posterior density approximation such as the Laplace's method \citep{zero84} or posterior simulation such as the Markov chain Monte Carlo (McMC) methods \citep{barn12} are most popular tools to obtain these Bayesian estimates. For instance, \citet{shsm11,shsm12,shao12,shgi14} first obtain McMC samples of the $\beta$-parameters and then substitute these samples of parameters into the expression for BMD to obtain a sample of estimated BMDs. Then, they use the median or the mean of the McMC sample of estimated BMDs as potential alternatives for their Bayesian estimate. \citet{wang11} first use McMC to obtain the Bayesian estimate of the $\beta$-parameters in the model, and then substitute these parameter estimates into their expression for BMD to obtain the estimate of BMD. \citet{whba09a} first find the posterior mean of the $\beta$-parameters through McMC, and then substitute these Bayesian estimates of the parameters into their expression for BMD to find the Bayes estimate of BMD. \citet{faes06} find a Bayesian estimate of BMD by inverting the posterior mean of the estimated extra risk function, whose parameters are estimated via the Gibbs sampler (a special case of the McMC method). \citet{moib06} first estimate the relative risk associated with the exposure concentration, and then substitute to find the estimated BMD. \citet{held04} first obtains Monte Carlo samples of the additional risk, and then obtains the posterior median of the additional risk, after which he finds the estimated BMD by inverting the posterior median of the additional risk. \citet{chen10} first finds estimates of the parameters via Laplace's method and iterative reweighted least square (IRLS) and then substitutes into his expression for $\mbox{ED}_{100\rho}$ to find the estimate of ED. \citet{kuco99} first obtain the posterior mean of the parameters and then substitute to find the estimated ED.
%
%In all these approaches, a BMDL or a credible interval for the ED is often obtained by first constructing a Monte Carlo sample of the estimated BMD or ED and then taking, for example, the lower $5^{\scriptsize \mbox{th}}$ percentile of the sample as the 95\% lower limit; see, e.g., \citet{shsm11, shsm12}, \citet{shao12}, \citet{nauf09}, \citet{whba09a}, \citet{moib06} and \citet{kuco99}. Some authors also obtain the BMDL by first obtaining an upper credible limit of the excess risk function and then inverting this to find the BMDL; see, e.g., \citet{faes06} and \citet{held04}. Other methods to find a BMDL include use of Fieller's theorem \citep{chen10}, obtaining an approximate bivariate distribution of the BMD via a two-dimensional smoothing kernel and then obtaining the contour \citep{lizh08} or using adaptive direction sampling and Hyndman's method \citep{hu08}.
%
%Hierarchical Bayesian modeling in quantitative risk assessment may also consider multiple responses or multiple agents. \citet{faes06} jointly consider the probability of fetal death, fetal malformation and fetal weight loss in studying the impact of Ethylene Glycol (EG) on the fetus. \citet{lizh08} and \citet{hu08} consider more than one agent in their respective studies. Besides fully parametric Bayesian methods, semi-parametric or non-parametric methods are also seen. For example, \citet{whba12} considered semi-parametric models for the dose response, incorporating a probit kernel and cubic B-splines. They placed normal priors on the basis-function coefficients and built dose-response monotonicity into their prior hierarchical constraints. Noteworthily, \citeauthor{whba12} built informative priors at $d$ = 0 to incorporate potential historical control information. \citet{guro13} described a nonparametric Bayesian model for the quantal setting, with beta/Dirichlet priors on pertinent probabilities related to their nonparametric construction. Hierarchical Bayesian modeling is also used to perform model averaging (a method to obtain a weighted-averaged estimator across a variety of models). For example, \citet{shsm11, shsm12, shao12, shgi14} and \citet{moib06} perform fully Bayesian model averaging (BMA) to obtain an averaged estimate of the BMD. Details for BMA will be discussed in Chapter \ref{Chap:BMA}.
%
%The goal of the dissertation is to obtain an improved estimation and calculation archetype for the BMD and for the BMDL, by applying various Bayesian strategies for incorporating direct prior information about the BMD via a reparameterized binomial likelihood. Chapter \ref{Chap:Repar} describes a suite of models and the reparameterizations for them. Chapter \ref{Chap:Bayesian Frame} develops a Bayesian framework using the reparameterized $R(d)$'s. Chapter \ref{Chap:BMA} discusses the issues of model adequacy/uncertainty and describes Bayesian model averaging that attempts to provide a more model-robust option for BMD/BMDL estimation. Chapter \ref{Chap:Example} follows with a real carcinogenicity data example. Chapter \ref{Chap:Simu} examines the characteristics of the Bayesian methodology and the performance of Bayesian model averaged BMDL estimates via a series of simulation studies. Chapter \ref{Chap:disc} concludes the dissertation.
