\chapter{Introduction}\label{Chap:Intro}

\section{Single-subject transcriptome analytics}\label{sec:nof1}

\indent \indent The bulk of the projects described below involve development of single-subject transcriptome (i.e. gene expression data) methodology for precision medicine \citep{Hamburg2010}. The motivation for single-subject biomedical methods is straightforward. Biology is exceeding complex - every organism is inherently unique. In the medical setting, demographics, genetics, and epigenetics (to name a few) all contribute to a patient's diagnosis, treatment, and prognosis \citep{Kern2012}. Often a patient is given a drug that has, on average, a high success rate for their medical condition, but inexplicably the patient fails to respond and suffers a poor outcome. What if we can make treatment decisions and predictions for that \emph{individual's} dysregulated cellular mechanisms? 

The applied problem of single-subject studies is intriguing, but statisticians are trained to make inferences on population parameters, based on (often independently) sampled observations. Numerous philosophical and technical issues arise when dealing with only a single patient's data. It is worth noting that Fisher's famous tea-tasting experiment \citep{Fisher1935} can be viewed as a \emph{single-subject} study. That being said, applications for statistical \emph{single-subject inference} have created a demand for developing this unexplored area \citep{Bacchetti2011}. From a statistical research standpoint, \emph{single-subject inference} poses many open problems that involve statistical topics. These include multivariate statistics, computing, machine learning, high-dimensional data analysis, small sample analysis, paired-sample statistics, and data visualization.

\section{N-of-1-\emph{pathways} framework}\label{sec:nof1pathways}

\indent \indent A central theme of my dissertation research concerns the ``statistical component'' of the single-subject framework, N-of-1-\emph{pathways} \citep{Gardeux2014}. The basic goal is to discover single-subject dysregulated genetic pathways from gene expression data. Two paired samples are obtained from the same patient, one from a baseline and one from a case condition (e.g., blood before and after treatment, tumor and normal tissue, etc.). Messenger-RNA (mRNA) expression is then quantified from the samples using, for example, RNA-sequencing \citep{Wang2009}. Next the workflow takes an informatic turn - the mRNAs are annotated to gene sets (pathways) that are functionally related via knowledgebases, such as Gene Ontology \citep{Ashburner2000}. This is in effect a dimension-reduction technique that also provides mechanistic interpretation \citep{Mooney2015}. Indeed single biomarkers, mRNA or otherwise, have been notoriously difficult to use in creating therapies or classifiers \citep{Kern2012}. The last step in the framework is the statistical component. The differential expression within each pathway is quantified and tests of hypotheses are conducted. We coined the phrase \emph{differentially expressed pathway} (DEP) to describe when a null hypothesis of equal expression is rejected (as an analog to the well-known DEG; differentially expressed gene). 

\section{Article-based dissertation and content organization}\label{sec:org}
\indent \indent This dissertation serves as a unifying document of three related articles developed during my dissertation research. The body chapters contain a brief summary of each project with an emphasis on my individual contribution and the contribution of the work to both statistical science and the biomedical domain. The concluding chapter provides an overall summary and future directions. Following the references are appendices containing the complete, unaltered articles and their supplemental documents that were described in the preceding chapters.
