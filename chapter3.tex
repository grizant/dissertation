\chapter{Accounting for inter-gene correlation in N-of-1-\emph{pathways}} \label{Chap:ct}

The MD formulation (Section \ref{sec:md}; \cite{Schissler2015}) described above appears to hold some promise. However, only half of the initial goal was achieved. The other major shortcoming of the Wilcoxon approach, besides eliminating the loss of information from ranking, is the assumption of inter-gene independence in the calculation of the P-value. Genes are known to be co-expressed, especially within a curated gene set (\cite{Tamayo2016}) and, moreover, when measurements are obtained from a single subject. Correlation in the observations often leads to poor standard error (SE) estimates and positive correlation may drastically inflate false positive errors (due to underestimation of the SE). This observation leads naturally to study the impact of inter-gene correlation in the N-of-1-\emph{pathways} framework.

I recently submitted a paper describing an improvement of the statistical component of N-of-1-\emph{pathways}. This work is geared towards a statistical audience. Importantly, the article introduces a method that accounts for inter-gene correlation and yields satisfactory false positive rates with non-trivial co-expression in the pathway. In my previous works, I typically simulated mRNA expression via independent negative binomial or Poisson modeling assumptions. Here, I used copulas (\cite{Genest2007,Yan2007}) to create multivariate distributions that simulate more authentic pathway expression. Preliminary results are promising and represent a significant advance in the statistical theory of \emph{single-subject inference} in the context gene expression analytics.

- refer to Appendix A \\
- critical picture\\
- my contribution to the project, mention last table as precursor to CTCs\\ 
- contribution to the field\\
- Connect to MD
