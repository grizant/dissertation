\chapter{Accounting for inter-gene correlation in N-of-1-\emph{pathways}} \label{Chap:ct}

\indent \indent The MD formulation (Appendix \ref{App:A}; \cite{Schissler2015}) described above appears to hold some promise. However, only half of the initial goal was achieved. The other major shortcoming of the Wilcoxon approach, besides eliminating the loss of information from ranking, is the assumption of inter-gene independence in the calculation of the P-value. Genes are known to be co-expressed, especially within a curated gene set (\cite{Tamayo2016}) and, moreover, when measurements are obtained from a single subject. Inter-observation correlation often leads to poor standard error (SE) estimates and positive correlation may drastically inflate false positive errors (due to underestimation of the SE). This observation leads naturally to study the impact of inter-gene correlation in the N-of-1-\emph{pathways} framework.

Appendix \ref{App:C} contains a manuscript\footnote{Under review at \emph{Statistical Methods in Medical Research}} describing an improvement of the statistical component of N-of-1-\emph{pathways}. Importantly, the article introduces the first single-subject, gene set testing methodology that accounts for inter-gene correlation while yielding satisfactory false positive rates with non-trivial co-expression in the pathway. In my previous works, I typically simulated mRNA expression via independent negative binomial or Poisson modeling assumptions. Here, I used copulas (\cite{Genest2007,Yan2007}) to create multivariate distributions that simulate more authentic pathway expression. Notably, the method presented in Appendix \ref{App:C} provides a P-value from a reference $t$ distribution for the average log fold change within pathways. There is no mention of the MD formulation in the manuscript, but it is easy to see that MD adjustment is simply a constant and the $t$ construction holds for MD pathway scores as well as the described average log fold change.

This article was primarily conducted by myself with the guidance of Dr. WW Piegorsch. I conceived the copula approach to simulating pathways. Many proposed methods were experimented with and discounted (including dependent central limit theorems, exotic bootstrapping procedures, and regression). I eventually stumbled across the robust variance estimator for cluster-correlated data given by Williams (\cite{Williams2000}) and developed informatic approach of infusing context-specific, inter-gene correlation into the methodology. Then Dr. WW Piegorsch and I developed the clustering algorithm to induce the cluster structure within pathways. I subsequently derived the $t$ distribution under assumptions provided in Appendix \ref{App:C}. Dr. YA Lussier performed the biomedical evaluation of the results.
